\noindent
\vspace{5pt}
\rule{4.5in}{0.015in}\\
\noindent
{\LARGE \texttt{CMobotGroup::motionWait()}\index{CMobotGroup::motionWait()}}\\
%\phantomsection
\addcontentsline{toc}{section}{motionWait()}

\noindent
{\bf Synopsis}
\vspace{-8pt}
\begin{verbatim}
#include <mobot.h>
int CMobotGroup::motionWait();
\end{verbatim}

\noindent
{\bf Purpose}\\
Wait for a preprogrammed mobotic motion to finish.\\

\noindent
{\bf Return Value}\\
The function returns 0 on success and non-zero otherwise.\\

\noindent
{\bf Description}\\
This function is used to wait for a preprogrammed motion to finish. Functions such as
\texttt{motionInchwormLeftNB()} and \texttt{moveForwardNB()} do not wait for the motion to finish
moving before continuing. The
\texttt{motionWait()} function is used to wait for
preprogrammed motions to complete. See Section \ref{sec:preprogrammed_motions} for 
a list of all preprogrammed mobotic motions.\\

\noindent
{\bf Example}\\
See the sample program in Section \ref{sec:democode} on page \pageref{sec:democode}.
\noindent

\noindent
{\bf See Also}\\
\texttt{motionWait(), moveJointWait()}

%\CPlot::\DataThreeD(), \CPlot::\DataFile(), \CPlot::\Plotting(), \plotxy().\\
