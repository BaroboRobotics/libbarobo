\noindent
\vspace{5pt}
\rule{4.5in}{0.015in}\\
\noindent
{\LARGE \texttt{CMobot::resetToZero()}\index{CMobot::resetToZero()}}\\
{\LARGE \texttt{CMobot::resetToZeroNB()}\index{CMobot::resetToZeroNB()}}\\
%\phantomsection
\addcontentsline{toc}{section}{resetToZero()}
\addcontentsline{toc}{section}{resetToZeroNB()}

\noindent
{\bf Synopsis}
\vspace{-8pt}
\begin{verbatim}
#include <mobot.h>
int CMobot::resetToZero();
int CMobot::resetToZeroNB();
\end{verbatim}

\noindent
{\bf Purpose}\\
Move all of the joints of a mobot to their zero position.\\

\noindent
{\bf Return Value}\\
The function returns 0 on success and non-zero otherwise.\\

\noindent
{\bf Parameters}\\
None.\\

\noindent
{\bf Description}\\
\vspace{-12pt}
\begin{quote}
{\bf CMobot::resetToZero()}\\
This function moves all of the joints of a mobot to their zero position. If the
joint is currently multiple rotations away from its zero position, this
function will reset the joint revolutions and move the joint directly to the
zero position.

{\bf CMobot::resetToZeroNB()}\\
This function moves all of the joints of a mobot to their zero position.

The function \texttt{resetToZeroNB()} is the non-blocking version of
the \texttt{resetToZero()} function, which means that the function will return
immediately and the physical mobot motion will occur asynchronously. For
more details on blocking and non-blocking functions, please refer to 
Section \ref{sec:blocking} on page \pageref{sec:blocking}.\\
\end{quote}

\noindent
{\bf Example}\\
Please see the demo at Section \ref{sec:democode} on page \pageref{sec:democode}.\\
\noindent

\noindent
{\bf See Also}\\

%\CPlot::\DataThreeD(), \CPlot::\DataFile(), \CPlot::\Plotting(), \plotxy().\\
