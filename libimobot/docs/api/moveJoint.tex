\noindent
\vspace{5pt}
\rule{6.5in}{0.015in}
\noindent
{\LARGE \texttt{moveJoint()}\index{moveJoint()}}\\
\phantomsection
\addcontentsline{toc}{section}{moveJoint()}

\noindent
{\bf Synopsis}\\
\begin{verbatim}
#include "imobot.h"
int CiMobot::moveJoint(unsigned short id, double angle);
\end{verbatim}

\noindent
{\bf Purpose}\\
Move a joint on the iMobot from its current postion.\\

\noindent
{\bf Return Value}\\
The function returns 0 on success and non-zero otherwise.\\

\noindent
{\bf Parameters}
\vspace{-0.1in}
\begin{description}
\item               
\begin{tabular}{p{10 mm}p{145 mm}}
\texttt{id} & The joint number to move. \\
\texttt{angle} & The desired angle to move in degrees.
\end{tabular}
\end{description}

\noindent
{\bf Description}\\
This function instructs the iMobot to move a joint a certain angle from its current position.
For instance, if a joint is currently at 45 degrees, a command to move the joint by -20 degrees will cause the robot to pose the joint to $45-20 = 35$ degrees.
Multiple successive calls to this function may be used to move multiple joints
simultaneously. The \texttt{waitMotor()} or \texttt{moveWait()} functions should
be used if the program should wait for motions to complete before continuing. \\

\noindent
{\bf Example}\\
\noindent

\noindent
{\bf See Also}\\
\texttt{moveWait(), waitMotor()}

%\CPlot::\DataThreeD(), \CPlot::\DataFile(), \CPlot::\Plotting(), \plotxy().\\
