% Mobile-C Library 

%%%%%%%%%%%%%%%%%%%%%%%%%%%%%%%%%%%%%%%%%%%%%%%%%%%%%%%%%%%%%%%%%%%%%%
% Preamble {{{
%\documentclass[11pt]{article}
\documentclass[11pt]{report}
\usepackage{varioref}
\usepackage{times,verbatim,fancyheadings,makeidx}
\usepackage{moreverb}
%\usepackage{psfig}
\usepackage[pdftex]{hyperref}
\usepackage{hypcap}
\usepackage{fullpage}
\usepackage{amssymb,amsmath}
\usepackage{graphicx}
\usepackage{program}
%\headrulewidth 0.0pt
%\hoffset=-0.0625in
%\voffset=0pt
\setlength{\textheight}{9in}
\setlength{\textwidth}{6.5in}
\topmargin=0.05in
\makeindex
% }}} Preamble
%%%%%%%%%%%%%%%%%%%%%%%%%%%%%%%%%%%%%%%%%%%%%%%%%%%%%%%%%%%%%%%%%%%%%%

%%%%%%%%%%%%%%%%%%%%%%%%%%%%%%%%%%%%%%%%%%%%%%%%%%%%%%%%%%%%%%%%%%%%%%
% Title Page {{{
\begin{document}
\thispagestyle{empty}
\begin{center}
%\includegraphics[width=1.8in]{figure/mobilec_logo.png}


\vspace{0.5in}
{\Huge\sf\bf iMobot Programming API} \\
\vspace{2.0in}
{\large\sf\bf\today}
%September 20, 2007
\end{center}

\pagebreak
% }}} Title Page
%%%%%%%%%%%%%%%%%%%%%%%%%%%%%%%%%%%%%%%%%%%%%%%%%%%%%%%%%%%%%%%%%%%%%%

%%%%%%%%%%%%%%%%%%%%%%%%%%%%%%%%%%%%%%%%%%%%%%%%%%%%%%%%%%%%%%%%%%%%%%
% Abstract {{{
%\phantomsection
%\addcontentsline{toc}{chapter}{Abstract}
\begin{abstract} 
This library implements control functions for controlling an iMobot
robotic module.

\end{abstract}
\pagebreak
% Abstract }}}
%%%%%%%%%%%%%%%%%%%%%%%%%%%%%%%%%%%%%%%%%%%%%%%%%%%%%%%%%%%%%%%%%%%%%%

%%%%%%%%%%%%%%%%%%%%%%%%%%%%%%%%%%%%%%%%%%%%%%%%%%%%%%%%%%%%%%%%%%%%%%
% Table of Contents {{{
\pagenumbering{roman}
\setcounter{page}{1}
\tableofcontents
\pagebreak
% }}} Table of Contents
%%%%%%%%%%%%%%%%%%%%%%%%%%%%%%%%%%%%%%%%%%%%%%%%%%%%%%%%%%%%%%%%%%%%%%

%%%%%%%%%%%%%%%%%%%%%%%%%%%%%%%%%%%%%%%%%%%%%%%%%%%%%%%%%%%%%%%%%%%%%%
% Part 1 {{{
\pagenumbering{arabic}
\setcounter{page}{1}
\pagebreak
% }}} Part 1 
%%%%%%%%%%%%%%%%%%%%%%%%%%%%%%%%%%%%%%%%%%%%%%%%%%%%%%%%%%%%%%%%%%%%%%

%%%%%%%%%%%%%%%%%%%%%%%%%%%%%%%%%%%%%%%%%%%%%%%%%%%%%%%%%%%%%%%%%%%%%%
% Introduction {{{
%\pagenumbering{arabic}
%\setcounter{page}{1}
%\pagestyle{fancy}
%\chapter{Introduction}
%\pagebreak
% }}} Introduction
%%%%%%%%%%%%%%%%%%%%%%%%%%%%%%%%%%%%%%%%%%%%%%%%%%%%%%%%%%%%%%%%%%%%%%

%%%%%%%%%%%%%%%%%%%%%%%%%%%%%%%%%%%%%%%%%%%%%%%%%%%%%%%%%%%%%%%%%%%%%%
% iMobot Library Installation {{{
\chapter{iMobot Library Package Contents}
The contents of the package should include a header file named
\texttt{imobot.h}, a static library named \texttt{libimobot.a},
this documentation, a Makefile, and a test program named \texttt{simple.cpp}.

\chapter{Installation}
\begin{enumerate}
\item Copy the package archive onto the iMobot. One way to do this is to use the
built in web browser.
  \begin{enumerate}
  \item Use a VNC client to connect to the imobot. The default IP address of the iMobot is 192.168.0.123.
  \item Log in as yourself with your password.
  \item Left click on the background to open the menu.
  \item Click on the item \texttt{Applications -> Internet -> Midori}, which is the built in web-browser.
  \item Go to the address: http://snake.engr.ucdavis.edu/~dko/
  \item Download the package, named \texttt{libimobot-X.X.zip}, where the \texttt{X}'s represent the latest version number.
  \end{enumerate}
\item Open a terminal by entering the menu and selecting \texttt{Applications -> Accessories -> xterm}.
\item In the terminal, traverse to the directory where you downloaded the zip
file. Use the command \texttt{ls} to list the files and directories in your
current directory, \texttt{cd <dir\_name>} to go to an existing directory, \texttt{pwd} to print
your current directory, and \texttt{cd ..} to go to the parent directory.
\item Unzip the package with the command \texttt{unzip libimobot-X.X.zip}, replacing the X's with the version number that you downloaded.
\item Enter the unzipped directory, \texttt{cd libimobot-X.X.zip}. 
\item Compile the test program with the command \texttt{make}.
\end{enumerate}

\chapter{Compiling Your Applications}
The following steps present a simple way to get your program up and running. In
the following steps, the text \texttt{<PROGRAM>} should be replaced with the
name of your program. All commands should be executed in a terminal.
\begin{enumerate}
\item Create a new directory for your project. \texttt{mkdir <PROGRAM>}
\item Enter the new directory. \texttt{cd <PROGRAM>}
\item Copy the iMobot library and header file into the directory. \texttt{"cp
<LIBIMOBOT\_PATH>/libimobot.a <LIBIMOBOT\_PATH>/imobot.h ."} Make sure to
replace the string \texttt{<LIBIMOBOT\_PATH>} with the full path to the
libimobot-X.X directory.
\item Create your robotic control program. \texttt{chide <PROGRAM>.cpp}
\item Once you have written your program, compile it with the command \texttt{g++ <PROGRAM>.cpp -limobot -o <PROGRAM>} 
\end{enumerate}


%%%%%%%%%%%%%%%%%%%%%%%%%%%%%%%%%%%%%%%%%%%%%%%%%%%%%%%%%%%%%%%%%%%%%%
% SUB: Overview of Sample Application Programs {{{
\section{Overview of Sample Application Programs}
The following program is a simple program which moves some joints on the iMobot
before initializing the robot to listen for incoming Bluetooth commands.
\subsection{simple.cpp \label{subsec:simple.cpp}}
\listinginput{1}{../Demos/simple_cpp/simple.cpp}
\subsection{Explanation of simple.cpp}
\begin{itemize}
\item Lines 8-10 declare some local variables that are used throughout the program.
\item Line 10 declares the \texttt{robot} object, which represents the
capabilities of the iMobot module. This object contains various member
functions which may be executed by the user.
\item Lines 13-16 send commands to the iMobot to move all the joints to their
zero position.
\item Line 17 causes the program to wait until all the joints have stopped moving. 
\item Lines 20 and 21 instruct the robot to rotate joints 2 and 3 to rotate 90
degrees. Note that the joint numbers start at 0, so joints 2 and 3 are the
third and fourth joints, respectively. 
\item Lines 23 and 24 cause the program to wait until the third and fourth
joints have stopped moving.
\item Lines 26 and 27 instruct the robot to turn the third and fourth joints
back to their original zero position.
\item Lines 28 and 29 cause the program to wait until the third and fourth
joints have stopped moving.
\item Lines 33 and 34 start the Bluetooth listening service on channel 20,
which listens for Bluetooth remote commands and controls the robot accordingly. 
\item Line 37 terminates the robot control.
\end{itemize}

%%%%%%%%%%%%%%%%%%%%%%%%%%%%%%%%%%%%%%%%%%%%%%%%%%%%%%%%%%%%%%%%%%%%%%
% SUB: Execution of Sample Applications {{{
\section{Execution of Sample Applications}
%%%%%%%%%%%%%%%%%%%%%%%%%%%%%%%%%%%%%%%%%%%%%%%%%%%%%%%%%%%%%%%%%%%%%%
% Appendix {{{
\appendix
\chapter{libimobot API}
\lhead{libimobot API Documentation}
\noindent
The header file {\bf libimobot.h} defines all the data types, macros 
and function prototypes for the iMobot API library. The header file
declares a class called \texttt{CiMobot} which contains member functions which
may be used to control the robot.

\begin{table}[!hp]
\capstart
\begin{center}
\caption{CiMobot Member Functions.}
\begin{tabular}{p{58 mm}p{97 mm}}
%\begin{tabular}{ll}
\hline
Function & Description \\
\hline
%\texttt{pose()} \dotfill & Pose multiple joints of the iMobot. \\
\texttt{CiMobot()} \dotfill & The CiMobot constructor function. This function
is called automatically and should not be called explicitly. \\
\texttt{\textasciitilde CiMobot()} \dotfill & The CiMobot destructor function. This function
is called automatically and should not be called explicitly. \\
& \\
\texttt{getJointAngle()} \dotfill & Gets a joint's angle. \\
\texttt{getJointAngles()} \dotfill & Gets all joint angles. \\
\texttt{getMotorSpeed()} \dotfill & Gets a motor's speed. \\
\texttt{initListenerBluetooth()} \dotfill & Initialize the Bluetooth module to
listen for incoming commands. \\
\texttt{isBusy()} \dotfill & See if the robot is currently moving. \\
\texttt{listenerMainLoop()} \dotfill & Main execution loop for the Bluetooth listener. \\
\texttt{moveJoint()} \dotfill & Move a joint of the imobot by some amount from its current location. \\
\texttt{moveWait()} \dotfill & Wait until all motors have stopped moving. \\
\texttt{poseJoint()} \dotfill & Pose a single joint of the iMobot. \\
\texttt{setMotorDirection()} \dotfill & Set the motor direction of a motor. Set
to "0" for automatic direction, "1" for forward, and "2" for reverse. \\
\texttt{setMotorSpeed()} \dotfill & Sets a motor's speed. \\
\texttt{stop()} \dotfill & Stop all currently executing motions of the iMobot. \\
\texttt{terminate()} \dotfill & Terminate the robotic control. \\
\texttt{waitMotor()} \dotfill & Wait until the specified motor has stopped moving. \\
\hline
\end{tabular}
\end{center}
\label{mobilec_api_cbinary}
\end{table}

\newpage
\noindent
\vspace{5pt}
\rule{4.5in}{0.015in}\\
\noindent
{\LARGE \texttt{CMobot::getJointAngle()}\index{getJointAngle()}}\\
%\phantomsection
\addcontentsline{toc}{section}{getJointAngle()}

\noindent
{\bf Synopsis}\\
\begin{verbatim}
#include <mobot.h>
int CMobot::getJointAngle(mobotJointId_t id, double &position);
\end{verbatim}

\noindent
{\bf Purpose}\\
Connect to a remote MoBot via Bluetooth.\\

\noindent
{\bf Return Value}\\
The function returns 0 on success and non-zero otherwise.\\

\noindent
{\bf Parameters}\\
\vspace{-0.1in}
\begin{description}
\item               
\begin{tabular}{p{15 mm}p{105 mm}}
\texttt{id} & The joint number to wait for. This is an enumerated type 
discussed in Section \ref{sec:mobotJointId_t} on page
\pageref{sec:mobotJointId_t}.\\
\texttt{position} & A variable to store the current position of the MoBot
motor. The contents of this variable will be overwritten with a value that
represents the motor's angle in degrees.  \\
\end{tabular}
\end{description}

\noindent
{\bf Description}\\
This function gets the current motor position of a MoBot's motor. The
position returned is in units of degrees and is accurate to roughly $\pm0.1$
degrees. \\

\noindent
{\bf Example}\\
\noindent

\noindent
{\bf See Also}\\
\texttt{connectWithAddress()}

%\CPlot::\DataThreeD(), \CPlot::\DataFile(), \CPlot::\Plotting(), \plotxy().\\

\pagebreak
\noindent
\vspace{5pt}
\rule{4.5in}{0.015in}\\
\noindent
{\LARGE \texttt{CMobot::getJointAngles()}\index{CMobot::getJointAngles()}}\\
{\LARGE \texttt{CMobot::getJointAnglesAbs()}\index{CMobot::getJointAnglesAbs()}}\\
%\phantomsection
\addcontentsline{toc}{section}{getJointAngles()}

\noindent
{\bf Synopsis}
\vspace{-8pt}
\begin{verbatim}
#include <mobot.h>
int CMobot::getJointAngles(
    double &angle1,
    double &angle2,
    double &angle3,
    double &angle4);
int CMobot::getJointAnglesAbs(
    double &angle1,
    double &angle2,
    double &angle3,
    double &angle4);
\end{verbatim}

\noindent
{\bf Purpose}\\
Retrieve a Mobot's current joint angles.\\

\noindent
{\bf Return Value}\\
The function returns 0 on success and non-zero otherwise.\\

\noindent
{\bf Parameters}\\
\vspace{-0.1in}
\begin{description}
\item               
\begin{tabular}{p{15 mm}p{145 mm}}
\texttt{angle1} & A variable to store the current angle of the mobot
motor. The contents of this variable will be overwritten with a value that
represents the motor's angle in degrees.  \\
\texttt{angle2} & ...  \\
\texttt{angle3} & ...  \\
\texttt{angle4} & ...  \\
\end{tabular}
\end{description}

\noindent
{\bf Description}\\
This function gets the current motor angles of a Mobot's motors. The
angle returned is in units of degrees and is accurate to roughly $\pm0.17$
degrees. 

The function \texttt{getJointAngles()} always returns an angle between -180 and
+180 degrees. The \texttt{getJointAnglesAbs()} function, however, gets the total
angle the joint has turned since the mobot has been powered on. For instance, 
if the faceplate joint 1 has been rotated two full rotations after initial power up,
  the function \texttt{getJointAngles()} will report that the joint is at angle 0,
  whereas the function \texttt{getJointAnglesAbs()} will report that the joint
  angle is 720 degrees.
\\

\noindent
{\bf Example}\\
\noindent

\noindent
{\bf See Also}\\

%\CPlot::\DataThreeD(), \CPlot::\DataFile(), \CPlot::\Plotting(), \plotxy().\\

\pagebreak
\input{api/getMotorSpeed}
\pagebreak
\noindent
\vspace{5pt}
\rule{6.5in}{0.015in}
\noindent
{\LARGE \texttt{CiMobot::initListenerBluetooth()}\index{initListenerBluetooth()}}\\
\phantomsection
\addcontentsline{toc}{section}{initListenerBluetooth()}

\noindent
{\bf Synopsis}\\
\begin{verbatim}
#include <imobot.h>
int CiMobot::initListenerBluetooth(int channel);
\end{verbatim}

\noindent
{\bf Purpose}\\
Initialize the bluetooth listening service to process incoming bluetooth commands.\\

\noindent
{\bf Return Value}\\
The function returns 0 on success and non-zero otherwise.\\

\noindent
{\bf Parameters}
\vspace{-0.1in}
\begin{description}
\item               
\begin{tabular}{p{20 mm}p{145 mm}}
\texttt{channel} & The bluetooth channel to listen on. The default value is "20". \\
\end{tabular}
\end{description}

\noindent
{\bf Description}\\
This function initialized the iMobot's bluetooth listening service so that it may process incoming Bluetooth commands. If this function is not called, the iMobot will not listen to any bluetooth commands.

\noindent
{\bf Example}\\
See the sample program in Section \ref{subsec:simple.cpp} on page \pageref{subsec:simple.cpp}.
\noindent

\noindent
{\bf See Also}\\
\texttt{moveWait(), waitMotor()}

%\CPlot::\DataThreeD(), \CPlot::\DataFile(), \CPlot::\Plotting(), \plotxy().\\

\pagebreak
\noindent
\vspace{5pt}
\rule{6.5in}{0.015in}
\noindent
{\LARGE \texttt{CiMobot::isBusy()}\index{isBusy()}}\\
\phantomsection
\addcontentsline{toc}{section}{isBusy()}

\noindent
{\bf Synopsis}\\
\begin{verbatim}
#include <imobot.h>
int CiMobot::isBusy();
\end{verbatim}

The usage of this function is identical to the
\texttt{CMobot::isBusy()} function for the MoBot. Please refer
to the MoBot documentation for \texttt{CMobot::isBusy} for
detailed usage documentation.


\pagebreak
\noindent
\vspace{5pt}
\rule{6.5in}{0.015in}
\noindent
{\LARGE \texttt{CiMobot::listenerMainLoop()}\index{listenerMainLoop()}}\\
\phantomsection
\addcontentsline{toc}{section}{listenerMainLoop()}

\noindent
{\bf Synopsis}\\
\begin{verbatim}
#include <imobot.h>
int CiMobot::listenerMainLoop();
\end{verbatim}

\noindent
{\bf Purpose}\\
Put the iMobot into Bluetooth listening mode.\\

\noindent
{\bf Return Value}\\
The function returns 0 on success and non-zero otherwise.\\

\noindent
{\bf Description}\\
This function puts the iMobot into Bluetooth listening mode. This function will
not return until it receives a Bluetooth "quit" command.

\noindent
{\bf Example}\\
See the sample program in Section \ref{subsec:simple.cpp} on page \pageref{subsec:simple.cpp}.
\noindent

\noindent
{\bf See Also}\\

%\CPlot::\DataThreeD(), \CPlot::\DataFile(), \CPlot::\Plotting(), \plotxy().\\

\pagebreak
\noindent
\vspace{5pt}
\rule{6.5in}{0.015in}
\noindent
{\LARGE \texttt{CiMobot::moveJoint()}\index{moveJoint()}}\\
{\LARGE \texttt{CiMobot::moveJointNB()}\index{moveJointNB()}}\\
\phantomsection
\addcontentsline{toc}{section}{moveJoint()}
\addcontentsline{toc}{section}{moveJointNB()}

\noindent
{\bf Synopsis}\\
\begin{verbatim}
#include <imobot.h>
int CiMobot::moveJoint( double angle1, 
                        double angle2, 
                        double angle3, 
                        double angle4);

int CiMobot::moveJointNB( double angle1, 
                          double angle2, 
                          double angle3, 
                          double angle4);
\end{verbatim}

The usage of these functions are identical to the
\texttt{CMobot::moveJoint()} and \texttt{CMobot::moveJointNB()} functions for the MoBot.
Please refer to the MoBot documentation for \texttt{CMobot::moveJoint()} and
\texttt{CMobot::moveJointNB()} for
detailed usage documentation.


\pagebreak
\noindent
\vspace{5pt}
\rule{4.5in}{0.015in}\\
\noindent
{\LARGE \texttt{CMobot::moveWait()}\index{CMobot::moveWait()}}\\
%\phantomsection
\addcontentsline{toc}{section}{moveWait()}

\noindent
{\bf Synopsis}
\vspace{-8pt}
\begin{verbatim}
#include <mobot.h>
int CMobot::moveWait();
\end{verbatim}

\noindent
{\bf Purpose}\\
Wait for all joints to stop moving.\\

\noindent
{\bf Return Value}\\
The function returns 0 on success and non-zero otherwise.\\

\noindent
{\bf Description}\\
This function is used to wait for all joint motions to finish. Functions such as
\texttt{move()} and \texttt{moveTo()} do not wait for a joint to finish
moving before continuing to allow multiple joints to move at the same time. The
\texttt{moveWait()} function is used to wait for
mobotic motions to complete.

Please note that if this function is called after a motor has been commanded to
turn indefinitely, this function may never return and your program may hang.\\

\noindent
{\bf Example}\\
See the sample program in Section \ref{sec:democode} on page \pageref{sec:democode}.
\noindent

\noindent
{\bf See Also}\\
\texttt{moveWait(), moveJointWait()}

%\CPlot::\DataThreeD(), \CPlot::\DataFile(), \CPlot::\Plotting(), \plotxy().\\

\pagebreak
\noindent
\vspace{5pt}
\rule{6.5in}{0.015in}
\noindent
{\LARGE \texttt{poseJoint()}\index{poseJoint()}}\\
\phantomsection
\addcontentsline{toc}{section}{poseJoint()}

\noindent
{\bf Synopsis}\\
\begin{verbatim}
#include "imobot.h"
int CiMobot::poseJoint(unsigned short id, double angle);
\end{verbatim}

\noindent
{\bf Purpose}\\
Pose a joint on the iMobot.\\

\noindent
{\bf Return Value}\\
The function returns 0 on success and non-zero otherwise.\\

\noindent
{\bf Parameters}
\vspace{-0.1in}
\begin{description}
\item               
\begin{tabular}{p{10 mm}p{145 mm}}
\texttt{id} & The joint number to pose. \\
\texttt{angle} & The desired angle in degrees.
\end{tabular}
\end{description}

\noindent
{\bf Description}\\
This function instructs the iMobot to pose a joint to a certain angle. This
function does not wait for the motion to complete before returning. Thus,
multiple successive calls to this function may be used to pose multiple joints
simultaneously. The \texttt{waitMotor()} or \texttt{moveWait()} functions should
be used if the program should wait for motions to complete before continuing. \\

\noindent
{\bf Example}\\
See the sample program in Section \ref{subsec:simple.cpp} on page \pageref{subsec:simple.cpp}.
\noindent

\noindent
{\bf See Also}\\
\texttt{moveWait(), waitMotor()}

%\CPlot::\DataThreeD(), \CPlot::\DataFile(), \CPlot::\Plotting(), \plotxy().\\

\pagebreak
\noindent
\vspace{5pt}
\rule{4.5in}{0.015in}\\
\noindent
{\LARGE \texttt{CiMobotComms::setMotorDirection()}\index{setMotorDirection()}}\\
%\phantomsection
\addcontentsline{toc}{section}{setMotorDirection()}

\noindent
{\bf Synopsis}\\
\begin{verbatim}
#include <imobotcomms.h>
int CiMobotComms::setMotorDirection(int id, int direction);
\end{verbatim}

\noindent
{\bf Purpose}\\
Set's a motor's direction. In conjunction with \texttt{setMotorSpeed()}, this
function may be used to cause a motor to turn indefinitely.\\

\noindent
{\bf Return Value}\\
The function returns 0 on success and non-zero otherwise.\\

\noindent
{\bf Parameters}
\vspace{-0.1in}
\begin{description}
\item               
\begin{tabular}{p{20 mm}p{145 mm}}
\texttt{id} & The joint number to move. \\
\texttt{direction} & A value indicating the desired direction.
\end{tabular}
\end{description}

\noindent
{\bf Description}\\
This function is used to set a motor's turn direction. Possible values for the
direction are:
\begin{itemize}
\item 0: Automatic direction. This is the default setting. 
\item 1: Forward. If this value is used with a non-zero speed set using the
\texttt{setMotorSpeed()} function, the motor will turn forward indefinitely.
\time 2: Backward. Similar to "1", except the motor will spin backward.
\end{itemize}

\noindent
{\bf Example}\\
\noindent

\noindent
{\bf See Also}\\

%\CPlot::\DataThreeD(), \CPlot::\DataFile(), \CPlot::\Plotting(), \plotxy().\\

\pagebreak
\noindent
\vspace{5pt}
\rule{6.5in}{0.015in}
\noindent
{\LARGE \texttt{CiMobot::setMotorSpeed()}\index{setMotorSpeed()}}\\
\phantomsection
\addcontentsline{toc}{section}{setMotorSpeed()}

\noindent
{\bf Synopsis}\\
\begin{verbatim}
#include <imobot.h>
int CiMobot::setMotorSpeed(unsigned short id, unsigned short speed);
\end{verbatim}

\noindent
{\bf Purpose}\\
Get the speed of a joint on the iMobot.\\

\noindent
{\bf Return Value}\\
The function returns 0 on success and non-zero otherwise.\\

\noindent
{\bf Parameters}
\vspace{-0.1in}
\begin{description}
\item               
\begin{tabular}{p{10 mm}p{145 mm}}
\texttt{id} & The joint number to pose. \\
\texttt{speed} & An unsigned short variable indicating the requested speed.
\end{tabular}
\end{description}

\noindent
{\bf Description}\\
This function is used to set the speed of a joint of an iMobot. Valid speed
values range from 0 to 100.
\noindent
{\bf Example}\\
\noindent

\noindent
{\bf See Also}\\

%\CPlot::\DataThreeD(), \CPlot::\DataFile(), \CPlot::\Plotting(), \plotxy().\\

\pagebreak
\noindent
\vspace{5pt}
\rule{6.5in}{0.015in}
\noindent
{\LARGE \texttt{CiMobot::stop()}\index{stop()}}\\
\phantomsection
\addcontentsline{toc}{section}{stop()}

\noindent
{\bf Synopsis}\\
\begin{verbatim}
#include <imobot.h>
int CiMobot::stop();
\end{verbatim}

The usage of this function is identical to the
\texttt{CMobot::stop()} function for the MoBot.
Please refer to the MoBot documentation for \texttt{CMobot::stop()} for
detailed usage documentation.


\pagebreak
\noindent
\vspace{5pt}
\rule{6.5in}{0.015in}
\noindent
{\LARGE \texttt{terminate()}\index{terminate()}}\\
\phantomsection
\addcontentsline{toc}{section}{terminate()}

\noindent
{\bf Synopsis}\\
\begin{verbatim}
#include "imobot.h"
int CiMobot::terminate();
\end{verbatim}

\noindent
{\bf Purpose}\\
Terminate control of an iMobot.\\

\noindent
{\bf Return Value}\\
The function returns 0 on success and non-zero otherwise.\\

\noindent
{\bf Description}\\
This function releases control of the I2C bus, thus terminating communications
with the iMobot. Only a single program may control the iMobot at the same time. If multiple programs are running, the program that currently has control of the iMobot needs to call the \texttt{terminate()} function before other programs can control the iMobot.

\noindent
{\bf Example}\\
See the sample program in Section \ref{subsec:simple.cpp} on page \pageref{subsec:simple.cpp}.
\noindent

\noindent
{\bf See Also}\\

%\CPlot::\DataThreeD(), \CPlot::\DataFile(), \CPlot::\Plotting(), \plotxy().\\

\pagebreak
\noindent
\vspace{5pt}
\rule{4.5in}{0.015in}\\
\noindent
{\LARGE \texttt{CiMobotComms::waitMotor()}\index{waitMotor()}}\\
%\phantomsection
\addcontentsline{toc}{section}{waitMotor()}

\noindent
{\bf Synopsis}\\
\begin{verbatim}
#include <imobot.h>
int CiMobotComms::waitMotor(int id);
\end{verbatim}

\noindent
{\bf Purpose}\\
Wait for a joint to stop moving.\\

\noindent
{\bf Return Value}\\
The function returns 0 on success and non-zero otherwise.\\

\noindent
{\bf Parameters}
\vspace{-0.1in}
\begin{description}
\item               
\begin{tabular}{p{10 mm}p{145 mm}}
\texttt{id} & The joint number to wait for. \\
\end{tabular}
\end{description}

\noindent
{\bf Description}\\
This function is used to wait for a joint motion to finish. Functions such as
\texttt{poseJoint()} and \texttt{moveJoint()} do not wait for a joint to finish
moving before continuing to allow multiple joints to move at the same time. The
\texttt{waitMotor()} or \texttt{waitMotor()} functions are used to wait for
robotic motions to complete.

Please note that if this function is called after a motor has been commanded to
turn indefinitely, this function may never return and your program may hang.\\

\noindent
{\bf Example}\\
See the sample program in Section \ref{subsec:simple.cpp} on page \pageref{subsec:simple.cpp}.
\noindent

\noindent
{\bf See Also}\\
\texttt{moveWait()}

%\CPlot::\DataThreeD(), \CPlot::\DataFile(), \CPlot::\Plotting(), \plotxy().\\


\pagebreak
% }}} Appendix 
%%%%%%%%%%%%%%%%%%%%%%%%%%%%%%%%%%%%%%%%%%%%%%%%%%%%%%%%%%%%%%%%%%%%%%

%%%%%%%%%%%%%%%%%%%%%%%%%%%%%%%%%%%%%%%%%%%%%%%%%%%%%%%%%%%%%%%%%%%%%%
% Index {{{
\phantomsection
\addcontentsline{toc}{chapter}{Index}
\printindex
% }}} Index 
%%%%%%%%%%%%%%%%%%%%%%%%%%%%%%%%%%%%%%%%%%%%%%%%%%%%%%%%%%%%%%%%%%%%%%
\end{document}
