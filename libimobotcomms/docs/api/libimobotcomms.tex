%\lhead{libimobotcomms API Documentation}
\noindent
The header file {\bf libimobotcomms.h} defines all the data types, macros 
and function prototypes for the iMobot API library. The header file
declares a class called \texttt{CiMobotComms} which contains member functions which
may be used to control the robot.

\begin{table}[!hp]
%\capstart
\begin{center}
\caption{CiMobotComms Member Functions.}
\begin{tabular}{p{38 mm}p{77 mm}}
%\begin{tabular}{ll}
\hline
Function & Description \\
\hline
%\texttt{pose()} \dotfill & Pose multiple joints of the iMobot. \\
\texttt{CiMobotComms()} \dotfill & The CiMobotComms constructor function. This function
is called automatically and should not be called explicitly. This constructor will 
automatically try to connect with a paired iMobot using the \texttt{connect()} member
function. To specify an address to connect to, please use the
\texttt{CiMobotComms(const char address[], int channel);} constructor.\\
\texttt{CiMobotComms(const char address[], int channel);} \dotfill & 
An alternate constructor which may be used to explicitly specify the address of an iMobot 
to connect to. \\
\texttt{\textasciitilde CiMobotComms()} \dotfill & The CiMobotComms destructor function. This function
is called automatically and should not be called explicitly. \\
& \\
\texttt{connect()} \dotfill & Connect to a remote iMobot module. This function connects to an already-paired iMobot module in Microsoft Windows. This function does not currently work for non-Windows operating systems, such as Mac or Linux. For those operating systems, please use the \texttt{connectAddress()} function instead. \\
\texttt{connectAddress()} \dotfill & Connect to an iMobot module by specifying its Bluetooth address. \\
\texttt{disconnect()} \dotfill & Disconnect from an iMobot module. \\
\texttt{getJointAngle()} \dotfill & Gets a joint's angle. \\
\texttt{getJointDirection()} \dotfill & Gets a motor's currently assigned direction. \\
\texttt{getJointSpeed()} \dotfill & Gets a motor's speed. \\
\texttt{getJointState()} \dotfill & Gets a motor's current status. \\
\texttt{isConnected()} \dotfill & This function is used to check the connection to an iMobot. \\
\texttt{move()} \dotfill & Move all four joints of the iMobot by specified angles. \\
\texttt{moveTo()} \dotfill & Move all four joints of the iMobot to specified absolute angles. \\
\texttt{moveJointTo()} \dotfill & Set the desired motor position. \\
\texttt{moveJointWait()} \dotfill & Wait until the specified motor has stopped moving. \\
\texttt{moveWait()} \dotfill & Wait until all motors have stopped moving. \\
\texttt{pozeZero()} \dotfill & Instructs all motors to go to their zero positions. \\
\texttt{setJointDirection()} \dotfill & Set the motor direction of a motor. Set
to "0" for automatic direction, "1" for forward, and "2" for reverse. \\
\texttt{setJointSpeed()} \dotfill & Sets a motor's speed. \\
\texttt{stop()} \dotfill & Stop all currently executing motions of the iMobot. \\
\hline
Compound Motions & These are convenience functions of commonly used compound motions. \\
\hline
\texttt{motionInchwormLeft()} \dotfill & Inchworm gait towards the left. \\
\texttt{motionInchwormRight()} \dotfill & Inchworm gait towards the right. \\
\texttt{motionRollBackward()} \dotfill & Roll on the faceplates toward the backward direction. \\
\texttt{motionRollForward()} \dotfill & Roll on the faceplates forwards. \\
\texttt{motionStand()} \dotfill & Stand the iMobot up on its end. \\
\texttt{motionTurnLeft()} \dotfill & Rotate the iMobot counterclockwise. \\
\texttt{motionTurnRight()} \dotfill & Rotate the iMobot clockwise. \\
\hline
\end{tabular}
\end{center}
\label{mobilec_api_cbinary}
\end{table}

\newpage
\noindent
\vspace{5pt}
\rule{4.5in}{0.015in}\\
\noindent
{\LARGE \texttt{CMobot::connect()}\index{CMobot::connect()}}\\
%\phantomsection
\addcontentsline{toc}{section}{connect()}

\noindent
{\bf Synopsis}
\vspace{-8pt}
\begin{verbatim}
#include <mobot.h>
int CMobot::connect();
\end{verbatim}

\noindent
{\bf Purpose}\\
Connect to a remote mobot via Bluetooth.\\

\noindent
{\bf Return Value}\\
The function returns 0 on success and non-zero otherwise.\\

\noindent
{\bf Parameters}\\
None.\\

\noindent
{\bf Description}\\
This function is used to connect to a mobot. The function looks inside of a 
Barobo configuration file and connects to the first mobot listed in the file.
The configuration file may be created and/or modified using the Mobot Controller
Interface, and selecting the ``Mobot $\rightarrow$ Configure Mobot Bluetooth'' menu item.

\noindent
{\bf Example}\\
Please see the example in Section \ref{sec:democode} on page \pageref{sec:democode}.\\
\noindent

\noindent
{\bf See Also}\\
\texttt{connectWithBluetoothAddress(), disconnect()}\\

%\CPlot::\DataThreeD(), \CPlot::\DataFile(), \CPlot::\Plotting(), \plotxy().\\

\pagebreak
\input{api/connectAddress}
\pagebreak
\noindent
\vspace{5pt}
\rule{4.5in}{0.015in}\\
\noindent
{\LARGE \texttt{CMobot::disconnect()}\index{CMobot::disconnect()}}\\
%\phantomsection
\addcontentsline{toc}{section}{disconnect()}

\noindent
{\bf Synopsis}
\begin{verbatim}
#include <mobot.h>
int CMobot::disconnect();
\end{verbatim}

\noindent
{\bf Purpose}\\
Disconnect from a remote robot.\\

\noindent
{\bf Return Value}\\
The function returns 0 on success and non-zero otherwise.\\

\noindent
{\bf Parameters}\\
None.\\

\noindent
{\bf Description}\\
This function is used from disconnect to a robot. A call to this function is
not necessary before the termination of a program. It is only necessary if
another connection will be established within the same program at a later time.
\\

\noindent
{\bf Example}\\
\noindent

\noindent
{\bf See Also}\\
\texttt{connect(), connectWithAddress()}

%\CPlot::\DataThreeD(), \CPlot::\DataFile(), \CPlot::\Plotting(), \plotxy().\\

\pagebreak
\noindent
\vspace{5pt}
\rule{4.5in}{0.015in}\\
\noindent
{\LARGE \texttt{CMobot::getJointAngle()}\index{getJointAngle()}}\\
%\phantomsection
\addcontentsline{toc}{section}{getJointAngle()}

\noindent
{\bf Synopsis}\\
\begin{verbatim}
#include <mobot.h>
int CMobot::getJointAngle(mobotJointId_t id, double &position);
\end{verbatim}

\noindent
{\bf Purpose}\\
Connect to a remote MoBot via Bluetooth.\\

\noindent
{\bf Return Value}\\
The function returns 0 on success and non-zero otherwise.\\

\noindent
{\bf Parameters}\\
\vspace{-0.1in}
\begin{description}
\item               
\begin{tabular}{p{15 mm}p{105 mm}}
\texttt{id} & The joint number to wait for. This is an enumerated type 
discussed in Section \ref{sec:mobotJointId_t} on page
\pageref{sec:mobotJointId_t}.\\
\texttt{position} & A variable to store the current position of the MoBot
motor. The contents of this variable will be overwritten with a value that
represents the motor's angle in degrees.  \\
\end{tabular}
\end{description}

\noindent
{\bf Description}\\
This function gets the current motor position of a MoBot's motor. The
position returned is in units of degrees and is accurate to roughly $\pm0.1$
degrees. \\

\noindent
{\bf Example}\\
\noindent

\noindent
{\bf See Also}\\
\texttt{connectWithAddress()}

%\CPlot::\DataThreeD(), \CPlot::\DataFile(), \CPlot::\Plotting(), \plotxy().\\

\pagebreak
\noindent
\vspace{5pt}
\rule{4.5in}{0.015in}\\
\noindent
{\LARGE \texttt{CMobot::getJointDirection()}\index{getJointDirection()}}\\
%\phantomsection
\addcontentsline{toc}{section}{getJointDirection()}

\noindent
{\bf Synopsis}\\
\begin{verbatim}
#include <mobot.h>
int CMobot::getJointDirection(int id, int &direction);
\end{verbatim}

\noindent
{\bf Purpose}\\
Get the speed of a joint on the MoBot.\\

\noindent
{\bf Return Value}\\
The function returns 0 on success and non-zero otherwise.\\

\noindent
{\bf Parameters}
\vspace{-0.1in}
\begin{description}
\item               
\begin{tabular}{p{10 mm}p{145 mm}}
\texttt{id} & The joint number to pose. \\
\texttt{direction} & An integer variable. This variable will be overwritten
with the current speed of the joint.
\end{tabular}
\end{description}

\noindent
{\bf Description}\\
This function is used to retrieve the motor's direction status. The valid
status directions are
\begin{itemize}
\item 0: Automatic direction
\item 1: Forward direction
\item 2: Backward direction
\end{itemize}

\noindent
{\bf Example}\\
\noindent

\noindent
{\bf See Also}\\
\texttt{setJointDirection()}

%\CPlot::\DataThreeD(), \CPlot::\DataFile(), \CPlot::\Plotting(), \plotxy().\\

\pagebreak
\noindent
\vspace{5pt}
\rule{4.5in}{0.015in}\\
\noindent
{\LARGE \texttt{CiMobotComms::getJointSpeed()}\index{getJointSpeed()}}\\
%\phantomsection
\addcontentsline{toc}{section}{getJointSpeed()}

\noindent
{\bf Synopsis}\\
\begin{verbatim}
#include <imobot.h>
int CiMobotComms::getJointSpeed(int id, int &speed);
\end{verbatim}

\noindent
{\bf Purpose}\\
Get the speed of a joint on the iMobot.\\

\noindent
{\bf Return Value}\\
The function returns 0 on success and non-zero otherwise.\\

\noindent
{\bf Parameters}
\vspace{-0.1in}
\begin{description}
\item               
\begin{tabular}{p{10 mm}p{145 mm}}
\texttt{id} & The joint number to pose. \\
\texttt{speed} & The address of an unsigned short variable. This variable will be overwritten
with the current speed of the joint.
\end{tabular}
\end{description}

\noindent
{\bf Description}\\
This function is used to find the speed of a joint.  This is the speed at which the joint will move when given motion commands. The values should be between 0 and 100. \\

\noindent
{\bf Example}\\
\noindent

\noindent
{\bf See Also}\\

%\CPlot::\DataThreeD(), \CPlot::\DataFile(), \CPlot::\Plotting(), \plotxy().\\

\pagebreak
\noindent
\vspace{5pt}
\rule{4.5in}{0.015in}\\
\noindent
{\LARGE \texttt{CMobot::getJointState()}\index{CMobot::getJointState()}}\\
%\phantomsection
\addcontentsline{toc}{section}{getJointState()}

\noindent
{\bf Synopsis}
\vspace{-8pt}
\begin{verbatim}
#include <mobot.h>
int CMobot::getJointState(mobotJointId_t id, robotJointState_t &state);
\end{verbatim}

\noindent
{\bf Purpose}\\
Determine whether a motor is moving or not.\\

\noindent
{\bf Return Value}\\
The function returns 0 on success and non-zero otherwise.\\

\noindent
{\bf Parameters}
\vspace{-0.1in}
\begin{description}
\item               
\begin{tabular}{p{10 mm}p{145 mm}}
\texttt{id} & The joint number. This is an enumerated type 
discussed in Section \ref{sec:mobotJointId_t} on page
\pageref{sec:mobotJointId_t}.\\
\texttt{state} & An integer variable which will be overwritten with the current state of the motor. 
This is an enumerated type 
discussed in Section \ref{sec:robotJointState_t} on page
\pageref{sec:robotJointState_t}.
\end{tabular}
\end{description}

\noindent
{\bf Description}\\
This function is used to determine the current state of a motor. Valid states are listed below.
\\
\noindent
\begin{tabular}{p{1.75in}p{4.5in}} \hline 
Value & Description \\
\hline \\
\texttt{ROBOT\_NEUTRAL}& This value indicates that the joint is not moving and is not actuated. The joint is freely backdrivable. \\
\texttt{ROBOT\_FORWARD}& This value indicates that the joint is currently moving forward. \\
\texttt{ROBOT\_BACKWARD}& This value indicates that the joint is currently moving backward. \\
\texttt{ROBOT\_HOLD}& This value indicates that the joint is currently not moving and is holding its current position. The joint is not currently backdrivable. \\
\hline
\end{tabular}\\



\noindent
{\bf Example}\\
\noindent

\noindent
{\bf See Also}\\
\texttt{isMoving()}\\
%\CPlot::\DataThreeD(), \CPlot::\DataFile(), \CPlot::\Plotting(), \plotxy().\\

\pagebreak
\noindent
\vspace{5pt}
\rule{4.5in}{0.015in}\\
\noindent
{\LARGE \texttt{CMobot::isConnected()}\index{CMobot::isConnected()}}\\
%\phantomsection
\addcontentsline{toc}{section}{isConnected()}

\noindent
{\bf Synopsis}
\vspace{-8pt}
\begin{verbatim}
#include <mobot.h>
int CMobot::isConnected();
\end{verbatim}

\noindent
{\bf Purpose}\\
Check to see if currently connected to a remote mobot via Bluetooth.\\

\noindent
{\bf Return Value}\\
The function returns zero if it is not currently connected to a mobot or if an 
error has occured, or 1 if the mobot is connected.\\

\noindent
{\bf Parameters}\\
None.\\

\noindent
{\bf Description}\\
This function is used to check if the software is currently connected to
a mobot.\\

\noindent
{\bf Example}\\
\noindent

\noindent
{\bf See Also}\\
\texttt{connect(), disconnect()}\\
%\CPlot::\DataThreeD(), \CPlot::\DataFile(), \CPlot::\Plotting(), \plotxy().\\

\pagebreak
\noindent
\vspace{5pt}
\rule{4.5in}{0.015in}\\
\noindent
{\LARGE \texttt{CMobotGroup::motionInchwormLeft()}\index{CMobotGroup::motionInchwormLeft()}}\\
{\LARGE \texttt{CMobotGroup::motionInchwormLeftNB()}\index{CMobotGroup::motionInchwormLeftNB()}}\\
%\phantomsection
\addcontentsline{toc}{section}{motionInchwormLeft()}
\addcontentsline{toc}{section}{motionInchwormLeftNB()}

\noindent
{\bf Synopsis}
\vspace{-8pt}
\begin{verbatim}
#include <mobot.h>
int CMobotGroup::motionInchwormLeft(int num);
int CMobotGroup::motionInchwormLeftNB(int num);
\end{verbatim}

\noindent
{\bf Purpose}\\
Make all robots in the group perform the inch-worm gait to the left.\\

\noindent
{\bf Return Value}\\
The function returns 0 on success and non-zero otherwise.\\

\noindent
{\bf Parameters}\\
\vspace{-0.1in}
\begin{description}
\item               
\begin{tabular}{p{15 mm}p{145 mm}}
\texttt{num} & The number of times to perform the inchworm gait.\\
\end{tabular}
\end{description}

\noindent
{\bf Description}\\
\vspace{-12pt}
\begin{quote}
{\bf CMobot::motionInchwormLeft()}\\
This function causes the robots to perform a single cycle of the inchworm gait
to the left. 

{\bf CMobot::motionInchwormLeftNB()}\\
This function causes the robots to perform a single cycle of the inchworm gait
to the left. 

The function \texttt{motionInchwormLeft()} is blocking, and the function
will hang until the motion has finished. The alternative function, \texttt{motionInchwormLeftNB()} 
will return immediately, and the motion will execute asynchronously. \\
\end{quote}

\noindent
{\bf See Also}\\
\texttt{motionInchwormRight()}

%\CPlot::\DataThreeD(), \CPlot::\DataFile(), \CPlot::\Plotting(), \plotxy().\\

\pagebreak
\noindent
\vspace{5pt}
\rule{4.5in}{0.015in}\\
\noindent
{\LARGE \texttt{CMobot::motionInchwormRight()}\index{CMobot::motionInchwormRight()}}\\
{\LARGE \texttt{CMobot::motionInchwormRightNB()}\index{CMobot::motionInchwormRightNB()}}\\
%\phantomsection
\addcontentsline{toc}{section}{motionInchwormRight()}
\addcontentsline{toc}{section}{motionInchwormRightNB()}

\noindent
{\bf Synopsis}
\vspace{-8pt}
\begin{verbatim}
#include <mobot.h>
int CMobot::motionInchwormRight(int num);
int CMobot::motionInchwormRightNB(int num);
\end{verbatim}

\noindent
{\bf Purpose}\\
Perform the inch-worm gait to the right.\\

\noindent
{\bf Return Value}\\
The function returns 0 on success and non-zero otherwise.\\

\noindent
{\bf Parameters}\\
\vspace{-0.1in}
\begin{description}
\item               
\begin{tabular}{p{15 mm}p{145 mm}}
\texttt{num} & The number of times to perform the inchworm gait.\\
\end{tabular}
\end{description}

\noindent
{\bf Description}\\
\vspace{-12pt}
\begin{quote}
{\bf CMobot::motionInchwormRight()}\\
This function causes the robot to perform a single cycle of the inchworm gait
to the right. 

{\bf CMobot::motionInchwormRightNB()}\\
This function causes the robot to perform a single cycle of the inchworm gait
to the right. 

This function has both a blocking and non-blocking version.
The blocking version, \texttt{motionInchwormRight()}, will block until the
robot motion has completed. The non-blocking version, \texttt{motionInchwormRightNB()},
will return immediately, and the motion will be performed asynchronously.\\
\end{quote}

\noindent
{\bf See Also}\\
\texttt{motionInchwormLeft()}

%\CPlot::\DataThreeD(), \CPlot::\DataFile(), \CPlot::\Plotting(), \plotxy().\\

\pagebreak
\noindent
\vspace{5pt}
\rule{4.5in}{0.015in}\\
\noindent
{\LARGE \texttt{CMobot::motionRollBackward()}\index{motionRollBackward()}}\\
%\phantomsection
\addcontentsline{toc}{section}{motionRollBackward()}

\noindent
{\bf Synopsis}\\
\begin{verbatim}
#include <imobot.h>
int CMobot::motionRollBackward();
\end{verbatim}

\noindent
{\bf Purpose}\\
Use the faceplates as wheels to roll backward.\\

\noindent
{\bf Return Value}\\
The function returns 0 on success and non-zero otherwise.\\

\noindent
{\bf Parameters}\\
None.\\

\noindent
{\bf Description}\\
This function causes each of the faceplates to rotate 90 degrees to roll the
robot backward.\\

\noindent
{\bf See Also}\\
\texttt{motionRollBackward()}

%\CPlot::\DataThreeD(), \CPlot::\DataFile(), \CPlot::\Plotting(), \plotxy().\\

\pagebreak
\noindent
\vspace{5pt}
\rule{4.5in}{0.015in}\\
\noindent
{\LARGE \texttt{CMobotGroup::motionRollForward()}\index{CMobotGroup::motionRollForward()}}\\
{\LARGE \texttt{CMobotGroup::motionRollForwardNB()}\index{CMobotGroup::motionRollForwardNB()}}\\
%\phantomsection
\addcontentsline{toc}{section}{motionRollForward()}
\addcontentsline{toc}{section}{motionRollForwardNB()}

\noindent
{\bf Synopsis}
\vspace{-8pt}
\begin{verbatim}
#include <mobot.h>
int CMobotGroup::motionRollForward(double angle);
int CMobotGroup::motionRollForwardNB(double angle);
\end{verbatim}

\noindent
{\bf Purpose}\\
Use the faceplates as wheels to roll robots forward.\\

\noindent
{\bf Return Value}\\
The function returns 0 on success and non-zero otherwise.\\

\noindent
{\bf Parameters}\\
\vspace{-0.1in}
\begin{description}
\item               
\begin{tabular}{p{15 mm}p{145 mm}}
\texttt{angle} & The angle to turn the wheels, specified in degrees.\\
\end{tabular}
\end{description}

\noindent
{\bf Description}\\
\vspace{-12pt}
\begin{quote}
{\bf CMobot::motionRollForward()}\\
This function causes each of the faceplates to rotate 90 degrees to roll the
robots forward.

{\bf CMobot::motionRollForwardNB()}\\
This function causes each of the faceplates to rotate 90 degrees to roll the
robots forward.

This function has both a blocking and non-blocking version.
The blocking version, \texttt{motionRollForward()}, will block until the
robot motion has completed. The non-blocking version, \texttt{motionRollForwardNB()},
will return immediately, and the motion will be performed asynchronously.\\
\end{quote}

\noindent
{\bf See Also}\\
\texttt{motionRollBackward()}

%\CPlot::\DataThreeD(), \CPlot::\DataFile(), \CPlot::\Plotting(), \plotxy().\\

\pagebreak
\noindent
\vspace{5pt}
\rule{4.5in}{0.015in}\\
\noindent
{\LARGE \texttt{CMobot::motionStand()}\index{CMobot::motionStand()}}\\
{\LARGE \texttt{CMobot::motionStandNB()}\index{CMobot::motionStandNB()}}\\
%\phantomsection
\addcontentsline{toc}{section}{motionStand()}
\addcontentsline{toc}{section}{motionStandNB()}

\noindent
{\bf Synopsis}
\begin{verbatim}
#include <mobot.h>
int CMobot::motionStand();
int CMobot::motionStandNB();
\end{verbatim}

\noindent
{\bf Purpose}\\
Stand the robot up on a faceplate.\\

\noindent
{\bf Return Value}\\
The function returns 0 on success and non-zero otherwise.\\

\noindent
{\bf Parameters}\\
None.\\

\noindent
{\bf Description}\\
This function causes the robot to motionStand up into the camera platform.

This function has both a blocking and non-blocking version.
The blocking version, \texttt{motionStand()}, will block until the
robot motion has completed. The non-blocking version, \texttt{motionStandNB()},
will return immediately, and the motion will be performed asynchronously.\\

\noindent
{\bf See Also}\\

%\CPlot::\DataThreeD(), \CPlot::\DataFile(), \CPlot::\Plotting(), \plotxy().\\

\pagebreak
\noindent
\vspace{5pt}
\rule{4.5in}{0.015in}\\
\noindent
{\LARGE \texttt{CMobot::motionTurnLeft()}\index{CMobot::motionTurnLeft()}}\\
{\LARGE \texttt{CMobot::motionTurnLeftNB()}\index{CMobot::motionTurnLeftNB()}}\\
%\phantomsection
\addcontentsline{toc}{section}{motionTurnLeft()}
\addcontentsline{toc}{section}{motionTurnLeftNB()}

\noindent
{\bf Synopsis}
\vspace{-8pt}
\begin{verbatim}
#include <mobot.h>
int CMobot::motionTurnLeft(double angle);
int CMobot::motionTurnLeftNB(double angle);
\end{verbatim}

\noindent
{\bf Purpose}\\
Rotate the mobot using the faceplates as wheels.\\

\noindent
{\bf Return Value}\\
The function returns 0 on success and non-zero otherwise.\\

\noindent
{\bf Parameters}\\
\vspace{-0.1in}
\begin{description}
\item               
\begin{tabular}{p{10 mm}p{145 mm}}
\texttt{angle} & The angle in degrees to turn the wheels. The wheels will turn in opposite directions by the amount specified by this argument in order to turn the mobot to the left. \\
\end{tabular}
\end{description}

\noindent
{\bf Description}\\
\vspace{-12pt}
\begin{quote}
{\bf CMobot::motionTurnLeft()}\\
This function causes the mobot to rotate the faceplates in opposite directions
to cause the mobot to rotate counter-clockwise.

{\bf CMobot::motionTurnLeftNB()}\\
This function causes the mobot to rotate the faceplates in opposite directions
to cause the mobot to rotate counter-clockwise.

This function has both a blocking and non-blocking version.
The blocking version, \texttt{motionTurnLeft()}, will block until the
mobot motion has completed. The non-blocking version, \texttt{motionTurnLeftNB()},
will return immediately, and the motion will be performed asynchronously.\\
\end{quote}


\noindent
{\bf See Also}\\
\texttt{motionTurnRight()}

%\CPlot::\DataThreeD(), \CPlot::\DataFile(), \CPlot::\Plotting(), \plotxy().\\

\pagebreak
\noindent
\vspace{5pt}
\rule{4.5in}{0.015in}\\
\noindent
{\LARGE \texttt{CMobot::motionTurnRight()}\index{motionTurnRight()}}\\
%\phantomsection
\addcontentsline{toc}{section}{motionTurnRight()}

\noindent
{\bf Synopsis}\\
\begin{verbatim}
#include <mobot.h>
int CMobot::motionTurnRight();
\end{verbatim}

\noindent
{\bf Purpose}\\
Rotate the iMobot using the faceplates as wheels.\\

\noindent
{\bf Return Value}\\
The function returns 0 on success and non-zero otherwise.\\

\noindent
{\bf Parameters}\\
None.\\

\noindent
{\bf Description}\\
This function causes the iMobot to rotate the faceplates in opposite directions
to cause the robot to rotate clockwise.\\

\noindent
{\bf See Also}\\
\texttt{motionTurnLeft()}

%\CPlot::\DataThreeD(), \CPlot::\DataFile(), \CPlot::\Plotting(), \plotxy().\\

\pagebreak
\noindent
\vspace{5pt}
\rule{6.5in}{0.015in}
\noindent
{\LARGE \texttt{CiMobot::move()}\index{move()}}\\
{\LARGE \texttt{CiMobot::moveNB()}\index{moveNB()}}\\
\phantomsection
\addcontentsline{toc}{section}{move()}
\addcontentsline{toc}{section}{moveNB()}

\noindent
{\bf Synopsis}\\
\begin{verbatim}
#include <imobot.h>
int CiMobot::move( double angle1, 
                   double angle2, 
                   double angle3, 
                   double angle4);

int CiMobot::moveNB( double angle1, 
                     double angle2, 
                     double angle3, 
                     double angle4);
\end{verbatim}

The usage of these functions are identical to the
\texttt{CMobot::move()} and \texttt{CMobot::moveNB()} functions for the MoBot.
Please refer to the MoBot documentation for \texttt{CMobot::move()} and
\texttt{CMobot::moveNB()} for
detailed usage documentation.


\pagebreak
\noindent
\vspace{5pt}
\rule{6.5in}{0.015in}
\noindent
{\LARGE \texttt{CiMobot::moveTo()}\index{moveTo()}}\\
{\LARGE \texttt{CiMobot::moveToNB()}\index{moveToNB()}}\\
\phantomsection
\addcontentsline{toc}{section}{moveTo()}
\addcontentsline{toc}{section}{moveToNB()}

\noindent
{\bf Synopsis}\\
\begin{verbatim}
#include <imobot.h>
int CiMobot::moveTo( double angle1, 
                     double angle2, 
                     double angle3, 
                     double angle4);

int CiMobot::moveToNB( double angle1, 
                       double angle2, 
                       double angle3, 
                       double angle4);
\end{verbatim}

The usage of these functions are identical to the
\texttt{CMobot::moveTo()} and \texttt{CMobot::moveToNB()} functions for the MoBot.
Please refer to the MoBot documentation for \texttt{CMobot::moveTo()} and
\texttt{CMobot::moveToNB()} for
detailed usage documentation.


\pagebreak
\noindent
\vspace{5pt}
\rule{6.5in}{0.015in}
\noindent
{\LARGE \texttt{CiMobot::moveJointTo()}\index{moveJointTo()}}\\
{\LARGE \texttt{CiMobot::moveJointToNB()}\index{moveJointToNB()}}\\
\phantomsection
\addcontentsline{toc}{section}{moveJointTo()}
\addcontentsline{toc}{section}{moveJointToNB()}

\noindent
{\bf Synopsis}\\
\begin{verbatim}
#include <imobot.h>
int CiMobot::moveJointTo( double angle1, 
                          double angle2, 
                          double angle3, 
                          double angle4);

int CiMobot::moveJointToNB( double angle1, 
                            double angle2, 
                            double angle3, 
                            double angle4);
\end{verbatim}

The usage of these functions are identical to the
\texttt{CMobot::moveJointTo()} and \texttt{CMobot::moveJointToNB()} functions for the MoBot.
Please refer to the MoBot documentation for \texttt{CMobot::moveJointTo()} and
\texttt{CMobot::moveJointToNB()} for
detailed usage documentation.


\pagebreak
\noindent
\vspace{5pt}
\rule{6.5in}{0.015in}
\noindent
{\LARGE \texttt{CiMobot::moveJointWait()}\index{moveJointWait()}}\\
\phantomsection
\addcontentsline{toc}{section}{moveJointWait()}

\noindent
{\bf Synopsis}\\
\begin{verbatim}
#include <imobot.h>
int CiMobot::moveJointWait(iMobotJointId_t id);
\end{verbatim}

The usage of this function is identical to the
\texttt{CMobot::moveJointWait()} function for the MoBot.
Please refer to the MoBot documentation for \texttt{CMobot::moveJointWait()} for
detailed usage documentation.


\pagebreak
\noindent
\vspace{5pt}
\rule{4.5in}{0.015in}\\
\noindent
{\LARGE \texttt{CMobot::moveWait()}\index{CMobot::moveWait()}}\\
%\phantomsection
\addcontentsline{toc}{section}{moveWait()}

\noindent
{\bf Synopsis}
\vspace{-8pt}
\begin{verbatim}
#include <mobot.h>
int CMobot::moveWait();
\end{verbatim}

\noindent
{\bf Purpose}\\
Wait for all joints to stop moving.\\

\noindent
{\bf Return Value}\\
The function returns 0 on success and non-zero otherwise.\\

\noindent
{\bf Description}\\
This function is used to wait for all joint motions to finish. Functions such as
\texttt{move()} and \texttt{moveTo()} do not wait for a joint to finish
moving before continuing to allow multiple joints to move at the same time. The
\texttt{moveWait()} function is used to wait for
mobotic motions to complete.

Please note that if this function is called after a motor has been commanded to
turn indefinitely, this function may never return and your program may hang.\\

\noindent
{\bf Example}\\
See the sample program in Section \ref{sec:democode} on page \pageref{sec:democode}.
\noindent

\noindent
{\bf See Also}\\
\texttt{moveWait(), moveJointWait()}

%\CPlot::\DataThreeD(), \CPlot::\DataFile(), \CPlot::\Plotting(), \plotxy().\\

\pagebreak
\input{api/moveZero}
\pagebreak
\noindent
\vspace{5pt}
\rule{4.5in}{0.015in}\\
\noindent
{\LARGE \texttt{CMobot::setJointDirection()}\index{CMobot::setJointDirection()}}\\
%\phantomsection
\addcontentsline{toc}{section}{setJointDirection()}

\noindent
{\bf Synopsis}\\
\begin{verbatim}
#include <mobot.h>
enum robot_motor_direction_e
{
  IMOBOT_JOINT_DIR_AUTO,
  IMOBOT_JOINT_DIR_FORWARD,
  IMOBOT_JOINT_DIR_BACKWARD
};
int CMobot::setJointDirection(int id, int direction);
\end{verbatim}

\noindent
{\bf Purpose}\\
Set's a motor's direction. In conjunction with \texttt{setJointSpeed()}, this
function may be used to cause a motor to turn indefinitely.\\

\noindent
{\bf Return Value}\\
The function returns 0 on success and non-zero otherwise.\\

\noindent
{\bf Parameters}
\vspace{-0.1in}
\begin{description}
\item               
\begin{tabular}{p{20 mm}p{145 mm}}
\texttt{id} & The joint number to move. \\
\texttt{direction} & A value indicating the desired direction.
\end{tabular}
\end{description}

\noindent
{\bf Description}\\
This function is used to set a motor's turn direction. Possible values for the
direction are:
\begin{itemize}
\item 0: Automatic direction. This is the default setting. 
\item 1: Forward. If this value is used with a non-zero speed set using the
\texttt{setJointSpeed()} function, the motor will turn forward indefinitely.
\item 2: Backward. Similar to "1", except the motor will spin backward.
\end{itemize}

\noindent
{\bf Example}\\
\noindent

\noindent
{\bf See Also}\\

%\CPlot::\DataThreeD(), \CPlot::\DataFile(), \CPlot::\Plotting(), \plotxy().\\

\pagebreak
\noindent
\vspace{5pt}
\rule{6.5in}{0.015in}
\noindent
{\LARGE \texttt{CiMobot::setJointSpeed()}\index{setJointSpeed()}}\\
\phantomsection
\addcontentsline{toc}{section}{setJointSpeed()}

\noindent
{\bf Synopsis}\\
\begin{verbatim}
#include <imobot.h>
int CiMobot::setJointSpeed(iMobotJointId_t id, double speed);
\end{verbatim}

The usage of this function is identical to the
\texttt{CMobot::setJointSpeed()} function for the MoBot.
Please refer to the MoBot documentation for \texttt{CMobot::setJointSpeed()} for
detailed usage documentation.


\pagebreak
\noindent
\vspace{5pt}
\rule{6.5in}{0.015in}
\noindent
{\LARGE \texttt{CiMobot::stop()}\index{stop()}}\\
\phantomsection
\addcontentsline{toc}{section}{stop()}

\noindent
{\bf Synopsis}\\
\begin{verbatim}
#include <imobot.h>
int CiMobot::stop();
\end{verbatim}

The usage of this function is identical to the
\texttt{CMobot::stop()} function for the MoBot.
Please refer to the MoBot documentation for \texttt{CMobot::stop()} for
detailed usage documentation.


