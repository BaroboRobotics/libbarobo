%\lhead{libimobotcomms API Documentation}
\noindent
The header file {\bf libimobotcomms.h} defines all the data types, macros 
and function prototypes for the iMobot API library. The header file
declares a class called \texttt{CiMobotComms} which contains member functions which
may be used to control the robot.

\begin{table}[!hp]
%\capstart
\begin{center}
\caption{CiMobotComms Member Functions.}
\begin{tabular}{p{38 mm}p{77 mm}}
%\begin{tabular}{ll}
\hline
Function & Description \\
\hline
%\texttt{pose()} \dotfill & Pose multiple joints of the iMobot. \\
\texttt{CiMobotComms()} \dotfill & The CiMobotComms constructor function. This function
is called automatically and should not be called explicitly. \\
\texttt{\textasciitilde CiMobotComms()} \dotfill & The CiMobotComms destructor function. This function
is called automatically and should not be called explicitly. \\
& \\
\texttt{connect()} \dotfill & Connect to a remote iMobot module. This function connects to an already-paired iMobot module in Microsoft Windows. This function does not currently work for non-Windows operating systems, such as Mac or Linux. For those operating systems, please use the \texttt{connectAddress()} function instead. \\
\texttt{connectAddress()} \dotfill & Connect to an iMobot module by specifying its Bluetooth address. \\
\texttt{disconnect()} \dotfill & Disconnect from an iMobot module. \\
\texttt{getMotorDirection()} \dotfill & Gets a motor's currently assigned direction. \\
\texttt{getMotorPosition()} \dotfill & Gets a joint's angle. \\
\texttt{getMotorSpeed()} \dotfill & Gets a motor's speed. \\
\texttt{getMotorState()} \dotfill & Gets a motor's current status. \\
\texttt{isConnected()} \dotfill & This function is used to check the connection to an iMobot. \\
\texttt{moveWait()} \dotfill & Wait until all motors have stopped moving. \\
\texttt{pozeZero()} \dotfill & Instructs all motors to go to their zero positions. \\
\texttt{setMotorDirection()} \dotfill & Set the motor direction of a motor. Set
to "0" for automatic direction, "1" for forward, and "2" for reverse. \\
\texttt{setMotorPosition()} \dotfill & Set the desired motor position. \\
\texttt{setMotorSpeed()} \dotfill & Sets a motor's speed. \\
\texttt{stop()} \dotfill & Stop all currently executing motions of the iMobot. \\
\texttt{waitMotor()} \dotfill & Wait until the specified motor has stopped moving. \\
\hline
Compound Motions & These are convenience functions of commonly used compound motions. \\
\hline
\texttt{inchLeft()} \dotfill & Inchworm gait towards the left. \\
\texttt{inchRight()} \dotfill & Inchworm gait towards the right. \\
\texttt{rollBackward()} \dotfill & Roll on the faceplates toward the backward direction. \\
\texttt{rollForward()} \dotfill & Roll on the faceplates forwards. \\
\texttt{stand()} \dotfill & Stand the iMobot up on its end. \\
\texttt{turnLeft()} \dotfill & Rotate the iMobot counterclockwise. \\
\texttt{turnRight()} \dotfill & Rotate the iMobot clockwise. \\
\hline
\end{tabular}
\end{center}
\label{mobilec_api_cbinary}
\end{table}

\newpage
\noindent
\vspace{5pt}
\rule{4.5in}{0.015in}\\
\noindent
{\LARGE \texttt{CMobot::connect()}\index{CMobot::connect()}}\\
%\phantomsection
\addcontentsline{toc}{section}{connect()}

\noindent
{\bf Synopsis}
\vspace{-8pt}
\begin{verbatim}
#include <mobot.h>
int CMobot::connect();
\end{verbatim}

\noindent
{\bf Purpose}\\
Connect to a remote mobot via Bluetooth.\\

\noindent
{\bf Return Value}\\
The function returns 0 on success and non-zero otherwise.\\

\noindent
{\bf Parameters}\\
None.\\

\noindent
{\bf Description}\\
This function is used to connect to a mobot. The function looks inside of a 
Barobo configuration file and connects to the first mobot listed in the file.
The configuration file may be created and/or modified using the Mobot Controller
Interface, and selecting the ``Mobot $\rightarrow$ Configure Mobot Bluetooth'' menu item.

\noindent
{\bf Example}\\
Please see the example in Section \ref{sec:democode} on page \pageref{sec:democode}.\\
\noindent

\noindent
{\bf See Also}\\
\texttt{connectWithBluetoothAddress(), disconnect()}\\

%\CPlot::\DataThreeD(), \CPlot::\DataFile(), \CPlot::\Plotting(), \plotxy().\\

\pagebreak
\input{api/connectAddress}
\pagebreak
\noindent
\vspace{5pt}
\rule{4.5in}{0.015in}\\
\noindent
{\LARGE \texttt{CMobot::disconnect()}\index{CMobot::disconnect()}}\\
%\phantomsection
\addcontentsline{toc}{section}{disconnect()}

\noindent
{\bf Synopsis}
\begin{verbatim}
#include <mobot.h>
int CMobot::disconnect();
\end{verbatim}

\noindent
{\bf Purpose}\\
Disconnect from a remote robot.\\

\noindent
{\bf Return Value}\\
The function returns 0 on success and non-zero otherwise.\\

\noindent
{\bf Parameters}\\
None.\\

\noindent
{\bf Description}\\
This function is used from disconnect to a robot. A call to this function is
not necessary before the termination of a program. It is only necessary if
another connection will be established within the same program at a later time.
\\

\noindent
{\bf Example}\\
\noindent

\noindent
{\bf See Also}\\
\texttt{connect(), connectWithAddress()}

%\CPlot::\DataThreeD(), \CPlot::\DataFile(), \CPlot::\Plotting(), \plotxy().\\

\pagebreak
\noindent
\vspace{5pt}
\rule{4.5in}{0.015in}\\
\noindent
{\LARGE \texttt{CiMobotComms::getMotorDirection()}\index{getMotorDirection()}}\\
%\phantomsection
\addcontentsline{toc}{section}{getMotorDirection()}

\noindent
{\bf Synopsis}\\
\begin{verbatim}
#include <imobotcomms.h>
int CiMobotComms::getMotorDirection(int id, int &direction);
\end{verbatim}

\noindent
{\bf Purpose}\\
Get the speed of a joint on the iMobot.\\

\noindent
{\bf Return Value}\\
The function returns 0 on success and non-zero otherwise.\\

\noindent
{\bf Parameters}
\vspace{-0.1in}
\begin{description}
\item               
\begin{tabular}{p{20 mm}p{145 mm}}
\texttt{id} & The joint number to pose. \\
\texttt{direction} & An integer variable. This variable will be overwritten
with the current speed of the joint.
\end{tabular}
\end{description}

\noindent
{\bf Description}\\
This function is used to retrieve the motor's direction status. The valid
status directions are
\begin{itemize}
\item 0: Automatic direction
\item 1: Forward direction
\item 2: Backward direction
\end{itemize}

\noindent
{\bf Example}\\
\noindent

\noindent
{\bf See Also}\\
\texttt{setMotorDirection()}

%\CPlot::\DataThreeD(), \CPlot::\DataFile(), \CPlot::\Plotting(), \plotxy().\\

\pagebreak
\input{api/getMotorPosition}
\pagebreak
\input{api/getMotorSpeed}
\pagebreak
\noindent
\vspace{5pt}
\rule{4.5in}{0.015in}\\
\noindent
{\LARGE \texttt{CiMobotComms::getMotorState()}\index{getMotorState()}}\\
%\phantomsection
\addcontentsline{toc}{section}{getMotorState()}

\noindent
{\bf Synopsis}\\
\begin{verbatim}
#include <imobot.h>
int CiMobotComms::getMotorState(int id, int &state);
\end{verbatim}

\noindent
{\bf Purpose}\\
Determine whether a motor is moving or not.\\

\noindent
{\bf Return Value}\\
The function returns 0 on success and non-zero otherwise.\\

\noindent
{\bf Parameters}
\vspace{-0.1in}
\begin{description}
\item               
\begin{tabular}{p{10 mm}p{145 mm}}
\texttt{id} & The joint number to pose. \\
\texttt{state} & An integer variable which will be overwritten with the current state of the motor. 
\end{tabular}
\end{description}

\noindent
{\bf Description}\\
This function is used to determine the current state of a motor. Valid states are:
\begin{itemize}
\item 0: The motor is idle.
\item 1: The motor is moving.
\item 2: The motor is heading towards a specified position.
\end{itemize}

\noindent
{\bf Example}\\
\noindent

\noindent
{\bf See Also}\\

%\CPlot::\DataThreeD(), \CPlot::\DataFile(), \CPlot::\Plotting(), \plotxy().\\

\pagebreak
\input{api/inchLeft}
\pagebreak
\input{api/inchRight}
\pagebreak
\noindent
\vspace{5pt}
\rule{4.5in}{0.015in}\\
\noindent
{\LARGE \texttt{CMobot::isConnected()}\index{CMobot::isConnected()}}\\
%\phantomsection
\addcontentsline{toc}{section}{isConnected()}

\noindent
{\bf Synopsis}
\vspace{-8pt}
\begin{verbatim}
#include <mobot.h>
int CMobot::isConnected();
\end{verbatim}

\noindent
{\bf Purpose}\\
Check to see if currently connected to a remote mobot via Bluetooth.\\

\noindent
{\bf Return Value}\\
The function returns zero if it is not currently connected to a mobot or if an 
error has occured, or 1 if the mobot is connected.\\

\noindent
{\bf Parameters}\\
None.\\

\noindent
{\bf Description}\\
This function is used to check if the software is currently connected to
a mobot.\\

\noindent
{\bf Example}\\
\noindent

\noindent
{\bf See Also}\\
\texttt{connect(), disconnect()}\\
%\CPlot::\DataThreeD(), \CPlot::\DataFile(), \CPlot::\Plotting(), \plotxy().\\

\pagebreak
\noindent
\vspace{5pt}
\rule{4.5in}{0.015in}\\
\noindent
{\LARGE \texttt{CMobot::moveWait()}\index{CMobot::moveWait()}}\\
%\phantomsection
\addcontentsline{toc}{section}{moveWait()}

\noindent
{\bf Synopsis}
\vspace{-8pt}
\begin{verbatim}
#include <mobot.h>
int CMobot::moveWait();
\end{verbatim}

\noindent
{\bf Purpose}\\
Wait for all joints to stop moving.\\

\noindent
{\bf Return Value}\\
The function returns 0 on success and non-zero otherwise.\\

\noindent
{\bf Description}\\
This function is used to wait for all joint motions to finish. Functions such as
\texttt{move()} and \texttt{moveTo()} do not wait for a joint to finish
moving before continuing to allow multiple joints to move at the same time. The
\texttt{moveWait()} function is used to wait for
mobotic motions to complete.

Please note that if this function is called after a motor has been commanded to
turn indefinitely, this function may never return and your program may hang.\\

\noindent
{\bf Example}\\
See the sample program in Section \ref{sec:democode} on page \pageref{sec:democode}.
\noindent

\noindent
{\bf See Also}\\
\texttt{moveWait(), moveJointWait()}

%\CPlot::\DataThreeD(), \CPlot::\DataFile(), \CPlot::\Plotting(), \plotxy().\\

\pagebreak
\noindent
\vspace{5pt}
\rule{4.5in}{0.015in}\\
\noindent
{\LARGE \texttt{CiMobotComms::poseZero()}\index{poseZero()}}\\
%\phantomsection
\addcontentsline{toc}{section}{poseZero()}

\noindent
{\bf Synopsis}\\
\begin{verbatim}
#include <imobot.h>
int CiMobotComms::poseZero();
\end{verbatim}

\noindent
{\bf Purpose}\\
Move all of the joints of an iMobot to their zero position.\\

\noindent
{\bf Return Value}\\
The function returns 0 on success and non-zero otherwise.\\

\noindent
{\bf Parameters}\\
None.\\

\noindent
{\bf Description}\\
This function moves all of the joints of an iMobot to their zero position.
Please note that this function is non-blocking and will return immediately. Use
this function in conjunction with the \texttt{moveWait()} function to block
until the movement completes.\\

\noindent
{\bf Example}\\
\noindent

\noindent
{\bf See Also}\\

%\CPlot::\DataThreeD(), \CPlot::\DataFile(), \CPlot::\Plotting(), \plotxy().\\

\pagebreak
\input{api/rollBackward}
\pagebreak
\noindent
\vspace{5pt}
\rule{4.5in}{0.015in}\\
\noindent
{\LARGE \texttt{CiMobotComms::rollForward()}\index{rollForward()}}\\
%\phantomsection
\addcontentsline{toc}{section}{rollForward()}

\noindent
{\bf Synopsis}\\
\begin{verbatim}
#include <imobot.h>
int CiMobotComms::rollForward();
\end{verbatim}

\noindent
{\bf Purpose}\\
Use the faceplates as wheels to roll forward.\\

\noindent
{\bf Return Value}\\
The function returns 0 on success and non-zero otherwise.\\

\noindent
{\bf Parameters}\\
None.\\

\noindent
{\bf Description}\\
This function causes each of the faceplates to rotate 90 degrees to roll the
robot forward.\\

\noindent
{\bf See Also}\\
\texttt{rollBackward()}

%\CPlot::\DataThreeD(), \CPlot::\DataFile(), \CPlot::\Plotting(), \plotxy().\\

\pagebreak
\noindent
\vspace{5pt}
\rule{4.5in}{0.015in}\\
\noindent
{\LARGE \texttt{CiMobotComms::setMotorDirection()}\index{setMotorDirection()}}\\
%\phantomsection
\addcontentsline{toc}{section}{setMotorDirection()}

\noindent
{\bf Synopsis}\\
\begin{verbatim}
#include <imobotcomms.h>
int CiMobotComms::setMotorDirection(int id, int direction);
\end{verbatim}

\noindent
{\bf Purpose}\\
Set's a motor's direction. In conjunction with \texttt{setMotorSpeed()}, this
function may be used to cause a motor to turn indefinitely.\\

\noindent
{\bf Return Value}\\
The function returns 0 on success and non-zero otherwise.\\

\noindent
{\bf Parameters}
\vspace{-0.1in}
\begin{description}
\item               
\begin{tabular}{p{20 mm}p{145 mm}}
\texttt{id} & The joint number to move. \\
\texttt{direction} & A value indicating the desired direction.
\end{tabular}
\end{description}

\noindent
{\bf Description}\\
This function is used to set a motor's turn direction. Possible values for the
direction are:
\begin{itemize}
\item 0: Automatic direction. This is the default setting. 
\item 1: Forward. If this value is used with a non-zero speed set using the
\texttt{setMotorSpeed()} function, the motor will turn forward indefinitely.
\time 2: Backward. Similar to "1", except the motor will spin backward.
\end{itemize}

\noindent
{\bf Example}\\
\noindent

\noindent
{\bf See Also}\\

%\CPlot::\DataThreeD(), \CPlot::\DataFile(), \CPlot::\Plotting(), \plotxy().\\

\pagebreak
\noindent
\vspace{5pt}
\rule{4.5in}{0.015in}\\
\noindent
{\LARGE \texttt{CiMobotComms::setMotorPosition()}\index{setMotorPosition()}}\\
%\phantomsection
\addcontentsline{toc}{section}{setMotorPosition()}

\noindent
{\bf Synopsis}\\
\begin{verbatim}
#include <imobot.h>
int CiMobotComms::setMotorPosition(int id, double position);
\end{verbatim}

\noindent
{\bf Purpose}\\
Connect to a remote iMobot via Bluetooth.\\

\noindent
{\bf Return Value}\\
The function returns 0 on success and non-zero otherwise.\\

\noindent
{\bf Parameters}\\
\vspace{-0.1in}
\begin{description}
\item               
\begin{tabular}{p{10 mm}p{145 mm}}
\texttt{id} & The joint number to wait for. \\
\texttt{position} & The absolute angle to move the motor to.  \\
\end{tabular}
\end{description}

\noindent
{\bf Description}\\
This function commands the motor to move to a position specified in degrees at
the current motor's speed. The current motor speed may be set with the
\texttt{setMotorSpeed()} member function. Please note that if the motor speed
is set to zero, the motor will not move after calling the
\texttt{setMotorPosition()} function. \\

\noindent
{\bf Example}\\
\noindent

\noindent
{\bf See Also}\\
\texttt{connectAddress()}

%\CPlot::\DataThreeD(), \CPlot::\DataFile(), \CPlot::\Plotting(), \plotxy().\\

\pagebreak
\noindent
\vspace{5pt}
\rule{6.5in}{0.015in}
\noindent
{\LARGE \texttt{CiMobot::setMotorSpeed()}\index{setMotorSpeed()}}\\
\phantomsection
\addcontentsline{toc}{section}{setMotorSpeed()}

\noindent
{\bf Synopsis}\\
\begin{verbatim}
#include <imobot.h>
int CiMobot::setMotorSpeed(unsigned short id, unsigned short speed);
\end{verbatim}

\noindent
{\bf Purpose}\\
Get the speed of a joint on the iMobot.\\

\noindent
{\bf Return Value}\\
The function returns 0 on success and non-zero otherwise.\\

\noindent
{\bf Parameters}
\vspace{-0.1in}
\begin{description}
\item               
\begin{tabular}{p{10 mm}p{145 mm}}
\texttt{id} & The joint number to pose. \\
\texttt{speed} & An unsigned short variable indicating the requested speed.
\end{tabular}
\end{description}

\noindent
{\bf Description}\\
This function is used to set the speed of a joint of an iMobot. Valid speed
values range from 0 to 100.
\noindent
{\bf Example}\\
\noindent

\noindent
{\bf See Also}\\

%\CPlot::\DataThreeD(), \CPlot::\DataFile(), \CPlot::\Plotting(), \plotxy().\\

\pagebreak
\noindent
\vspace{5pt}
\rule{4.5in}{0.015in}\\
\noindent
{\LARGE \texttt{CiMobotComms::stand()}\index{stand()}}\\
%\phantomsection
\addcontentsline{toc}{section}{stand()}

\noindent
{\bf Synopsis}\\
\begin{verbatim}
#include <imobot.h>
int CiMobotComms::stand();
\end{verbatim}

\noindent
{\bf Purpose}\\
Stand the robot up on a faceplate.\\

\noindent
{\bf Return Value}\\
The function returns 0 on success and non-zero otherwise.\\

\noindent
{\bf Parameters}\\
None.\\

\noindent
{\bf Description}\\
This function causes the robot to stand up into the camera platform.\\

\noindent
{\bf See Also}\\

%\CPlot::\DataThreeD(), \CPlot::\DataFile(), \CPlot::\Plotting(), \plotxy().\\

\pagebreak
\noindent
\vspace{5pt}
\rule{6.5in}{0.015in}
\noindent
{\LARGE \texttt{CiMobot::stop()}\index{stop()}}\\
\phantomsection
\addcontentsline{toc}{section}{stop()}

\noindent
{\bf Synopsis}\\
\begin{verbatim}
#include <imobot.h>
int CiMobot::stop();
\end{verbatim}

The usage of this function is identical to the
\texttt{CMobot::stop()} function for the MoBot.
Please refer to the MoBot documentation for \texttt{CMobot::stop()} for
detailed usage documentation.


\pagebreak
\noindent
\vspace{5pt}
\rule{4.5in}{0.015in}\\
\noindent
{\LARGE \texttt{CMobotGroup::turnLeft()}\index{CMobotGroup::turnLeft()}}\\
{\LARGE \texttt{CMobotGroup::turnLeftNB()}\index{CMobotGroup::turnLeftNB()}}\\
%\phantomsection
\addcontentsline{toc}{section}{turnLeft()}
\addcontentsline{toc}{section}{turnLeftNB()}

\noindent
{\bf Synopsis}
\vspace{-8pt}
\begin{verbatim}
#include <mobot.h>
int CMobotGroup::turnLeft(double angle);
int CMobotGroup::turnLeftNB(double angle);
\end{verbatim}

\noindent
{\bf Purpose}\\
Rotate the mobots using the faceplates as wheels.\\

\noindent
{\bf Return Value}\\
The function returns 0 on success and non-zero otherwise.\\

\noindent
{\bf Parameters}\\
\vspace{-0.1in}
\begin{description}
\item               
\begin{tabular}{p{10 mm}p{145 mm}}
\texttt{angle} & The angle in degrees to turn the wheels. The wheels will turn in opposite directions by the amount specified by this argument in order to turn the mobot to the left. \\
\end{tabular}
\end{description}

\noindent
{\bf Description}\\
This function causes the mobots to rotate the faceplates in opposite directions
to cause the mobot to rotate counter-clockwise.

This function has both a blocking and non-blocking version.
The blocking version, \texttt{turnLeft()}, will block until the
mobot motion has completed. The non-blocking version, \texttt{turnLeftNB()},
will return immediately, and the motion will be performed asynchronously.\\


\noindent
{\bf See Also}\\
\texttt{turnRight()}

%\CPlot::\DataThreeD(), \CPlot::\DataFile(), \CPlot::\Plotting(), \plotxy().\\

\pagebreak
\noindent
\vspace{5pt}
\rule{4.5in}{0.015in}\\
\noindent
{\LARGE \texttt{CiMobotComms::turnRight()}\index{turnRight()}}\\
%\phantomsection
\addcontentsline{toc}{section}{turnRight()}

\noindent
{\bf Synopsis}\\
\begin{verbatim}
#include <imobot.h>
int CiMobotComms::turnRight();
\end{verbatim}

\noindent
{\bf Purpose}\\
Rotate the iMobot using the faceplates as wheels.\\

\noindent
{\bf Return Value}\\
The function returns 0 on success and non-zero otherwise.\\

\noindent
{\bf Parameters}\\
None.\\

\noindent
{\bf Description}\\
This function causes the iMobot to rotate the faceplates in opposite directions
to cause the robot to rotate clockwise.\\

\noindent
{\bf See Also}\\
\texttt{turnLeft()}

%\CPlot::\DataThreeD(), \CPlot::\DataFile(), \CPlot::\Plotting(), \plotxy().\\

\pagebreak
\noindent
\vspace{5pt}
\rule{4.5in}{0.015in}\\
\noindent
{\LARGE \texttt{CiMobotComms::waitMotor()}\index{waitMotor()}}\\
%\phantomsection
\addcontentsline{toc}{section}{waitMotor()}

\noindent
{\bf Synopsis}\\
\begin{verbatim}
#include <imobot.h>
int CiMobotComms::waitMotor(int id);
\end{verbatim}

\noindent
{\bf Purpose}\\
Wait for a joint to stop moving.\\

\noindent
{\bf Return Value}\\
The function returns 0 on success and non-zero otherwise.\\

\noindent
{\bf Parameters}
\vspace{-0.1in}
\begin{description}
\item               
\begin{tabular}{p{10 mm}p{145 mm}}
\texttt{id} & The joint number to wait for. \\
\end{tabular}
\end{description}

\noindent
{\bf Description}\\
This function is used to wait for a joint motion to finish. Functions such as
\texttt{poseJoint()} and \texttt{moveJoint()} do not wait for a joint to finish
moving before continuing to allow multiple joints to move at the same time. The
\texttt{waitMotor()} or \texttt{waitMotor()} functions are used to wait for
robotic motions to complete.

Please note that if this function is called after a motor has been commanded to
turn indefinitely, this function may never return and your program may hang.\\

\noindent
{\bf Example}\\
See the sample program in Section \ref{subsec:simple.cpp} on page \pageref{subsec:simple.cpp}.
\noindent

\noindent
{\bf See Also}\\
\texttt{moveWait()}

%\CPlot::\DataThreeD(), \CPlot::\DataFile(), \CPlot::\Plotting(), \plotxy().\\

