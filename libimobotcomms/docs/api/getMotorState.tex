\noindent
\vspace{5pt}
\rule{4.5in}{0.015in}\\
\noindent
{\LARGE \texttt{CiMobotComms::getMotorState()}\index{getMotorState()}}\\
%\phantomsection
\addcontentsline{toc}{section}{getMotorState()}

\noindent
{\bf Synopsis}\\
\begin{verbatim}
#include <imobot.h>
int CiMobotComms::getMotorState(int id, int &state);
\end{verbatim}

\noindent
{\bf Purpose}\\
Determine whether a motor is moving or not.\\

\noindent
{\bf Return Value}\\
The function returns 0 on success and non-zero otherwise.\\

\noindent
{\bf Parameters}
\vspace{-0.1in}
\begin{description}
\item               
\begin{tabular}{p{10 mm}p{145 mm}}
\texttt{id} & The joint number to pose. \\
\texttt{state} & An integer variable which will be overwritten with the current state of the motor. 
\end{tabular}
\end{description}

\noindent
{\bf Description}\\
This function is used to determine the current state of a motor. Valid states are:
\begin{itemize}
\item 0: The motor is idle.
\item 1: The motor is moving.
\item 2: The motor is heading towards a specified position.
\end{itemize}

\noindent
{\bf Example}\\
\noindent

\noindent
{\bf See Also}\\

%\CPlot::\DataThreeD(), \CPlot::\DataFile(), \CPlot::\Plotting(), \plotxy().\\
