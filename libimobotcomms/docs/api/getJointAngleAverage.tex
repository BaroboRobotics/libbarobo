\noindent
\vspace{5pt}
\rule{4.5in}{0.015in}\\
\noindent
{\LARGE \texttt{CMobot::getJointAngleAverage()}\index{CMobot::getJointAngleAverage()}}\\
%\phantomsection
\addcontentsline{toc}{section}{getJointAngleAverage()}

\noindent
{\bf Synopsis}
\vspace{-8pt}
\begin{verbatim}
#include <mobot.h>
int CMobot::getJointAngleAverage(mobotJointId_t id, double &angle, int numReadings = 10);
\end{verbatim}

\noindent
{\bf Purpose}\\
Retrieve a mobot joint's current angle.\\

\noindent
{\bf Return Value}\\
The function returns 0 on success and non-zero otherwise.\\

\noindent
{\bf Parameters}\\
\vspace{-0.1in}
\begin{description}
\item               
\begin{tabular}{p{15 mm}p{145 mm}}
\texttt{id} & The joint number. This is an enumerated type 
discussed in Section \ref{sec:mobotJointId_t} on page
\pageref{sec:mobotJointId_t}.\\
\texttt{angle} & A variable to store the current angle of the mobot
motor. The contents of this variable will be overwritten with a value that
represents the motor's angle in degrees.  \\
\texttt{numReadings} & (Optional) The number of independent angle readings to measure 
and average. More readings taken may yield a more accurate angle measurement, but will also
take more time to perform. If this value is omitted, the default value of 10 readings is
used.
\end{tabular}
\end{description}

\noindent
{\bf Description}\\
This function gets the current motor angle of a Mobot's motor. In most cases, this
function may yield a more accurate value than the \texttt{getJointAngle()} function
because this function averages a number of separate angle readings. Averaging multiple 
readings may reduce the effect of noise on the angle measurements.
\\




\noindent
{\bf Example}\\
\noindent

\noindent
{\bf See Also}\\

%\CPlot::\DataThreeD(), \CPlot::\DataFile(), \CPlot::\Plotting(), \plotxy().\\
