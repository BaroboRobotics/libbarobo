\noindent
\vspace{5pt}
\rule{4.5in}{0.015in}\\
\noindent
{\LARGE \texttt{CMobot::getJointMaxSpeed()}\index{CMobot::getJointMaxSpeed()}}\\
%\phantomsection
\addcontentsline{toc}{section}{getJointMaxSpeed()}

\noindent
{\bf Synopsis}\\
\begin{verbatim}
#include <mobot.h>
int CMobot::getJointMaxSpeed(robotJointId_t id, double &speed);
\end{verbatim}

\noindent
{\bf Purpose}\\
Get the maximum speed of a joint on the robot.\\

\noindent
{\bf Return Value}\\
The function returns 0 on success and non-zero otherwise.\\

\noindent
{\bf Parameters}
\vspace{-0.1in}
\begin{description}
\item               
\begin{tabular}{p{10 mm}p{145 mm}}
\texttt{id} & The joint number. This is an enumerated type 
discussed in Section \ref{sec:robotJointId_t} on page
\pageref{sec:robotJointId_t}.\\
\texttt{speed} & A variable of type \texttt{double}. The value of this variable
will be overwritten with the maximum speed setting of the joint, which is 
in units of radians per second.
\end{tabular}
\end{description}

\noindent
{\bf Description}\\
This function is used to find the maximum speed setting of a joint.  This is
the maximum speed at which the joint will accept speed setting from the 
function \texttt{setJointSpeed()}. The values
are in units of radians per second.

\noindent
{\bf Example}\\
\noindent

\noindent
{\bf See Also}\\
\texttt{getJointSpeed(), getJointMaxSpeedRatio(), setJointSpeed(), setJointSpeedRatio()}

%\CPlot::\DataThreeD(), \CPlot::\DataFile(), \CPlot::\Plotting(), \plotxy().\\
