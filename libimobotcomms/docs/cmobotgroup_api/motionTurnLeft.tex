\noindent
\vspace{5pt}
\rule{4.5in}{0.015in}\\
\noindent
{\LARGE \texttt{CMobotGroup::motionTurnLeft()}\index{CMobotGroup::motionTurnLeft()}}\\
{\LARGE \texttt{CMobotGroup::motionTurnLeftNB()}\index{CMobotGroup::motionTurnLeftNB()}}\\
%\phantomsection
\addcontentsline{toc}{section}{motionTurnLeft()}
\addcontentsline{toc}{section}{motionTurnLeftNB()}

\noindent
{\bf Synopsis}
\vspace{-8pt}
\begin{verbatim}
#include <mobot.h>
int CMobotGroup::motionTurnLeft();
int CMobotGroup::motionTurnLeftNB();
\end{verbatim}

\noindent
{\bf Purpose}\\
Rotate the robots using the faceplates as wheels.\\

\noindent
{\bf Return Value}\\
The function returns 0 on success and non-zero otherwise.\\

\noindent
{\bf Parameters}\\
None.\\

\noindent
{\bf Description}\\
This function causes the robots to rotate the faceplates in opposite directions
to cause the robot to rotate counter-clockwise.

This function has both a blocking and non-blocking version.
The blocking version, \texttt{motionTurnLeft()}, will block until the
robot motion has completed. The non-blocking version, \texttt{motionTurnLeftNB()},
will return immediately, and the motion will be performed asynchronously.\\


\noindent
{\bf See Also}\\
\texttt{motionTurnRight()}

%\CPlot::\DataThreeD(), \CPlot::\DataFile(), \CPlot::\Plotting(), \plotxy().\\
