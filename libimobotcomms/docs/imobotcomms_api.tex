\documentclass{article}
\usepackage{graphicx}
\usepackage{wrapfig}
\usepackage{amsmath}
\usepackage{verbatim}
\usepackage{makeidx}

\title{Programming the MoBot}  
%\author{David Ko\\Mechanical and Aerospace Engineering}
%\date{\today} 
\makeindex

\begin{document}
\maketitle
\tableofcontents

\section{The \texttt{CMobot} MoBot Remote Control Library}
The \texttt{CMobot} library is a collection of functions geared towards
controlling the motors and reading sensor values of a MoBot module via the
Bluetooth wireless protocol. The functions are designed to be intuitive
and easy to use. Various functions are provided to control or obtain the speed,
direction, and position of the motors. The API includes C-style functions as well
as a C++ class called \texttt{CMobot} to facilitate C++ style api function
calls. 

This documentation introduces the basic computer setup required for controlling 
the MoBot, as well as several demo programs and a complete reference for all
API function provided with the \texttt{CMobot} library.

\section{\label{sec:pairing}Bluetooth Pairing with the MoBot}
To control the MoBot with the Bluetooth wireless protocol, the controlling 
computer must be equipped with Bluetooth. If the computer does not have
Bluetooth built-in, an external USB Bluetooth dongle may be used. The following
instructions are for a Windows 7 computer with built-in Bluetooth. The basic
process is the same for Windows XP and Vista, although the screenshots may
appear different. 

The first step is to open the Bluetooth applet by double-clicking the Bluetooth
icon in the applet tray normally found at the bottom right of the screen, as
shown highlighted in red in the following figure.

\begin{center}
\includegraphics[width=3in]{images/imobot_connect_1.png}
\end{center}

After double-clicking the icon, a new window will appear similar to the following.

\begin{center}
\includegraphics[width=5in]{images/imobot_connect_1a.png}
\end{center}

Next, click on the
button labeled "Add Wireless Device" towards the top of the window. This 
will bring up the following dialog.

\begin{center}
\includegraphics[width=5in]{images/imobot_connect_2.png}
\end{center}

This dialog shows a list of all Bluetooth wireless devices that are in range.
Among them should be the MoBot you wish to connect to. If the MoBot does not
appear on this list, please ensure that the MoBot is within 10 meters of the
connecting computer and that the MoBot is powered on. Double click on the icon
representing the MoBot to proceed. Once you have double-clicked the icon, the
following dialog box should appear.

\begin{center}
\includegraphics[width=5in]{images/imobot_connect_3.png}
\end{center}

Select the second option, labeled ``Enter the device's pairing code''. MoBot
modules come hard-coded with a default pairing code. When prompted for the
pairing code, enter ``1234''. Once the computer is paired with the MoBot,
the following dialog box may pop up asking to install drivers.

\begin{center}
\includegraphics[width=5in]{images/imobot_connect_4.png}
\end{center}

If the previously illustrated dialog box appears, just click the ``cancel''
button at the bottom right. No extra drivers are necessary for controlling the
MoBot module.

At this point, the following dialog should be shown.

\begin{center}
\includegraphics[width=5in]{images/imobot_connect_6.png}
\end{center}

The next step is to enable the MoBot control service. 
Double-click on the icon denoting the
MoBot module to bring up the following dialog:

\begin{center}
\includegraphics[width=5in]{images/imobot_connect_7.png}
\end{center}

Click on the tab labeled ``Services''

\begin{center}
\includegraphics[width=5in]{images/imobot_connect_8.png}
\end{center}

Ensure that the service titled ``MoBot Control'' is enabled. If it is not
enabled, click on the check-box to enable it. Click on the ``Ok'' button to
accept the changes and close the dialogs. The MoBot is now ready to be
controlled with the MoBotComms library.

\section{ The MoBot Remote Control Program }
\includegraphics[width=4.5in]{images/iMobot_controller_screenshot.png}

The preceding figure shows the MoBot remote control program. The
program displays a graphical interface which may be used to display
information about the MoBot's joint positions, and also control the
speeds and positions of the MoBot's joints. The interface is divided
up into six sections; three on the top half of the interface, and three on 
the bottom half. 

\subsection{The MoBot Diagram and ``Move To Zero'' Button}
The first section of the GUI located on the top left of the interface
displays a schematic diagram of the MoBot, displaying motor positions.
Underneath the diagram, there is a large button with the text 
``Move To Zero''. When clicked, this button will command the connected
MoBot to rotate all of its joints to a flat ``Zero'' position.

\subsection{Individual Joint Control}
The second section, located at the top-middle section of the interface,
is the ``Individual Joint Control'' section. These buttons command the
MoBot to move individual joints. When the up or down arrows are clicked,
the MoBot begins to move the corresponding joint in either the positive,
or negative direction. The joint will continue to move until the stop 
button, located between the up and down arrows, is clicked. 

\subsection{Rolling Control}
This section contains buttons for controlling the MoBot as a 
two wheeled mobile robot. The up and down buttons cause the MoBot to
roll forward or backward. The left and right buttons cause the MoBot 
to rotate towards the left, or towards the right. The stop button in the
middle causes the MoBot to stop where it is.

\subsection{Joint Speeds}
The ``Joint Speeds'' section, located at the bottom left of the interface,
displays and controls the current joint speeds of the MoBot. Joint speeds
are a value between 0 and 1, with 1 meaning maximum joint power, and 0
meaning zero joint power. The speed may be set by sliding the vertical 
sliders to the desired positions. 

\subsection{Joint Positions}
This section, located in the bottom-middle of the interface, is used to display
and control the positions of each of the four
joints of a MoBot. The joint positions are displayed in the numerical
text located above each vertical slider. The displayed joint positions are in
units of degrees.  There are two methods to control
the joints using this interface.

The first method of controlling the joints is by using the vertical sliders.
Each vertical slider's position represents a joint's angle. The sliders for the
two end joints vary from -180 degrees to 180 degrees, representing one complete
rotation. The angles for the two body joints vary from -90 to 90 degrees. When
the position of the slider is moved, the MoBot will move its joints to match the 
sliders. 

The second method for moving the joints is by entering the exact angles for the
joints. Below each of the four sliders lies a text entry box. Values in degrees
may be typed into each of the four entry boxes. When the button on the lower
right of the section labeled ``Move'' is clicked, the MoBot will move its joints
to match the values typed into the boxes. If no value is typed into a box, that 
joint will not move.

\subsection{Motions}
This section, located on the bottom right of the interface, contains a set of
preprogrammed motions for the MoBot. To execute a preprogrammed motion, simply
click on the name of the motion you wish to execute, and then click the button
labeled ``Play''.

\section{Sample Ch MoBot Programs}
To help the user become acquainted with the MoBot control programs, sample
programs will be presented to illustrate the basics and minimum requirements of
a MoBot control program. 

\subsection{Basics of a Ch MoBot Program}
The first demo presents a minimal program which connects to a MoBot and
moves some joints.

\subsubsection{\texttt{gettingStarted.ch} Source Code}
\verbatiminput{../demos/getting_started/gettingStarted.ch}

\subsubsection{\label{sec:democode}Demo Code for \texttt{gettingStarted.ch} Explained}
The beginning of every MoBot control program will include header files. Each
header file imports functions used for a number of tasks, such as printing
data onto the screen or controlling the MoBot. 

\begin{verbatim}
#include <mobot.h> // Required for MoBot control functions
\end{verbatim}

Next, we must initialize the C++ class used to control the MoBot. This line
initializes a new variable named \texttt{robot} which represents the remote
MoBot module which we wish to control. This special variable is actually an
instance of the \texttt{CMobot} class, which contains its own set of
functions called ``methods'' or ``member functions''.
\begin{verbatim}
CMobot robot;
\end{verbatim}

Next, we initialize two \texttt{double} type variables called 
``angle1'' and ``angle4'', which we will use to store some joint angles.
\begin{verbatim}
double angle1, angle4;
\end{verbatim}

The next line will connect our new variable, \texttt{robot}, to a
MoBot that has been previously paired with the computer.
\begin{verbatim}
robot.connect();
\end{verbatim}

Note that there are two common methods to connect to a remote MoBot. 
The most common method, demonstrated in the previous line of code, is
used to connect to a MoBot that is already paired to the computer. It
is also possible to connect to MoBots which are not paired with the 
computer. This method is necessary for connecting to multiple
MoBots simultaneously, as only a single MoBot may be paired with the
computer at a time. The second method uses the function
\texttt{connectWithAddress()}, and its default usage is as such:
\begin{verbatim}
robot.connectWithAddress("11:22:33:44:55:66", 20);
\end{verbatim}
The string \texttt{"11:22:33:44:55:66"} represents the Bluetooth address
of the MoBot, which must be known in advance. The number \texttt{20} 
represents the Bluetooth channel to connect to. Channel \texttt{20}
is the default channel MoBots listen on for incoming connections, but
may be set to other values.

The next line will use the \texttt{moveToZero} member function. The
\texttt{moveToZero} function causes the MoBot to move all of its motors to the
zero position.
\begin{verbatim}
robot.moveToZero();
\end{verbatim}

The next lines of code command joints 1 and 4 to rotate 90 degrees.
Joints 1 and 4 are the faceplates
of the MoBot which are sometimes used to act as "wheels".
\begin{verbatim}
angle1 = 90;
angle4 = 90;
robot.move(angle1, 0, 0, angle4);
\end{verbatim}

\subsection{Inchworm Gait Demo}
The next demo will demonstrate a simple gait known as the ``Inchworm'' gait.

\subsubsection{\texttt{inchworm.ch} Source Code}
\verbatiminput{../demos/inchworm/inchworm.ch}

\subsubsection{Demo Code for \texttt{inchworm.ch} Explained}
The first portion of the code is identical to the previous demo, and performs
the same function of declaring a MoBot variable and connecting to a 
paired MoBot.
\begin{verbatim}
#include <mobot.h>

CMobot robot;

/* Connect to the paired MoBot */
robot.connect();
\end{verbatim}

The next lines of code set the joint speeds for the two body joints, joints 
two and three, to 50\% speed. They are set to fify percent speed in order to 
slow the gait down in order to minimize slippage.

\begin{verbatim}
/* Set robot motors to speed of 0.50 */
robot.setJointSpeed(MOBOT_JOINT2, 0.50);
robot.setJointSpeed(MOBOT_JOINT3, 0.50);
\end{verbatim}

Next, we move the robot into a flat ``zero'' position.

\begin{verbatim}
/* Set the robot to "home" position, where all joint angles are 0 degrees. */
robot.moveToZero();
\end{verbatim}

Finally, we perform the actual inchworm gait. The inchworm gait is a gait defined
by a sequence of motions performed by the body joints. The motions are as such:
\begin{enumerate}
\item The first body joint, referred to as joint A, rotates towards the ground.
This drags the MoBot towards the direction of joint A.
\item The other body joint, joint B, rotates towards the ground. Since the center
of gravity is currently positioned over joint A, this causes the trailing body 
joint to slide toward joint A.
\item Joint A moves back to a flat position.
\item Joint B moves back to a flat position.
\item Repeat, if desired.
\end{enumerate}
The direction of travel depends on the selection of the initial body joint. In
the following code example, joint 2 is chosen as the initial body joint to move.
In this case, the MoBot will traverse towards joint 2. The entire gait is
encapsulated in a ``for'' loop which executes the entire gait four times.
\begin{verbatim}
/* Do the inchworm gait four times */
int i;
for(i = 0; i < 4; i++) {
  robot.moveJointTo(MOBOT_JOINT2, -45);
  robot.moveJointTo(MOBOT_JOINT3, 45);
  robot.moveJointTo(MOBOT_JOINT2, 0);
  robot.moveJointTo(MOBOT_JOINT3, 0);
}
\end{verbatim}

\section{Preprogrammed Motions}
The MoBot API contains functions for executing preprogrammed motions. The 
preprogrammed motions are motions which are commonly used for MoBot locomotion.
Following is a list of available functions and a brief description about
their effect on the MoBot.
\begin{itemize}
\item \texttt{motionInchwormLeft()}: This function causes the MoBot to perform
  the inchworm gait once, moving the MoBot towards its left.
\item \texttt{motionInchwormRight()}: This function causes the MoBot to perform
  the inchworm gait once, moving the MoBot towards its right.
\item \texttt{motionRollBackward()}: This function causes the MoBot to rotate
  its faceplates, using them as wheels to roll backward.
\item \texttt{motionRollForward()}: This function causes the MoBot to rotate
  its faceplates, using them as wheels to roll forward.
\item \texttt{motionStand()}: This function causes the MoBot to stand up onto a 
  faceplate, assuming the camera platform position.
\item \texttt{motionTurnLeft()}: Uses the MoBot's faceplates as wheels, turning
  them in opposite directions in order to rotate the MoBot towards its left.
\item \texttt{motionTurnRight()}: Uses the MoBot's faceplates as wheels, turning
  them in opposite directions in order to rotate the MoBot towards its right.
\end{itemize}

Note that all of the functions listed above are ``blocking'' functions, meaning
they will not return until the motion has completed. These functions also
have non-blocking equivalents which are discussed in Section
\ref{sec:blocking}.

\subsection{\texttt{stand.ch}: A Demo Using the \texttt{motionStand()} Preprogrammed
Motion}
This demo is a simple demonstration of the \texttt{motionStand()} member function.
\subsubsection{\texttt{stand.ch} Source Code}
\verbatiminput{../demos/stand/stand.ch}
\subsubsection{\texttt{stand.ch} Explained}
The source code for this demo is straightforward. After the initialization and 
connection as seen in previous demos, it executes the following line of
code:
\begin{verbatim}
robot.motionStand();
\end{verbatim}
This line of code causes the MoBot to perform a sequence of motions causing it to
stand up on a faceplate. The function is a blocking function and will wait until
the entire motion sequence is completed before continuing. There are also non-blocking
versions of the motion functions, documented in Section \ref{sec:blocking}.

\section{\label{sec:blocking}Blocking and Non-Blocking Functions}
All of the MoBot movement functions may be designated as either ``blocking'' 
functions or ``non-blocking'' functions. A blocking function is a function which
does not return while operations are being performed. For instance, the
\texttt{moveWait()} function is a blocking function. When called, the function
will hang, or ``block'', until all the joints have stopped moving. After all
joints have stopped moving, the \texttt{moveWait()} function will return, and 
the rest of the program will execute.

Furthermore, some functions have both a blocking version and a non-blocking
version. For these functions, the suffix \texttt{NB} denotes that the function
is non-blocking. For instance, the function \texttt{motionStand()} is blocking,
meaning the function will not return until the motion is completed, whereas
the function \texttt{motionStandNB()} is non-blocking, meaning the function
returns immediately and the robot performs the ``standing'' motion
asynchronously.

The function \texttt{move()} is an example of a non-blocking function. When
the \texttt{move()} function is called, the function immediately returns 
as the joints begin moving. Any lines of code following the call to 
\texttt{move()} will be executed even if the current motion is still in
progress. 

\subsection{List of Blocking Movement Functions}
\begin{itemize}
\item \texttt{move()}
\item \texttt{moveContinuousTime()}
\item \texttt{moveJointTo()}
\item \texttt{moveTo()}
\item \texttt{moveToZero()}
\item \texttt{moveJointWait()}
\item \texttt{moveWait()}
\item \texttt{motionInchwormLeft()}
\item \texttt{motionInchwormRight()}
\item \texttt{motionRollBackward()}
\item \texttt{motionRollForward()}
\item \texttt{motionStand()}
\item \texttt{motionTurnLeft()}
\item \texttt{motionTurnRight()}
\end{itemize}

\subsection{List of Non-Blocking Movement Functions}
\begin{itemize}
\item \texttt{moveNB()}
\item \texttt{moveContinuousNB()}
\item \texttt{moveJointToNB()}
\item \texttt{moveToNB()}
\item \texttt{moveToZeroNB()}
\item \texttt{motionInchwormLeftNB()}
\item \texttt{motionInchwormRightNB()}
\item \texttt{motionRollBackwardNB()}
\item \texttt{motionRollForwardNB()}
\item \texttt{motionStandNB()}
\item \texttt{motionTurnLeftNB()}
\item \texttt{motionTurnRightNB()}
\end{itemize}

\section{Controlling Multiple Modules}
The MoBot control software is designed to be able to control multiple modules
simultaneously. There are a couple important differences in the program 
which enable the control of multiple modules. A small demo program which
controls two modules simultaneously will first be presented, followed by
a detailed explanation of the program elements.

\subsection{Multiple Module Demo Program}
\verbatiminput{../demos/multiple_modules/multipleModules.cpp}

\subsection{Demo Explanation}
The first two lines of interest appear as such:
\begin{verbatim}
  CMobot robot1;
  CMobot robot2;
\end{verbatim}
These two lines declare two separate variables which will represent the
two separate MoBot modules. Next, we need to connect each variable to
a physically separate MoBot. This is done with the following lines.
\begin{verbatim}
  robot1.connectWithAddress("11:11:11:11:11:11", 20);
  robot2.connectWithAddress("22:22:22:22:22:22", 20);
\end{verbatim}
These lines connect the first variable, \texttt{robot1}, to the MoBot with
address \texttt{11:11:11:11:11:11}. When running this demo, this
demo address will need to be replaced with the actual address of the MoBot.
The second argument, \texttt{20}, is the channel to connect to. By default,
the MoBot will listen on channel 20 for incoming connections.

A similar process is done with \texttt{robot2}, causing it to connect
to a second MoBot with address \texttt{22:22:22:22:22:22}.

\begin{verbatim}
  robot1.moveToZeroNB();
  robot2.moveToZeroNB();
\end{verbatim}
These two lines command the two robots to move to their zero positions.
Note that these functions are non-blocking. This means that the
\texttt{moveToZeroNB()} function will return immediately, and will not
wait for the first robot to finish completing the motion before 
commanding the second robot to begin. In a normal program, this effectively
causes both robots to move to their zero positions simultaneously.

\begin{verbatim}
  robot1.moveWait();
  robot2.moveWait();
\end{verbatim}
Since the \texttt{moveToZeroNB()} functions are non-blocking, we would like
the program to wait until the motions are complete before continuing. By
calling \texttt{moveWait()} on both of the robots, we can be assured that
the robots have finished moving before the program continues.

\begin{verbatim}
  robot1.motionStandNB();
  robot2.motionStandNB();
  robot1.moveWait();
  robot2.moveWait();
\end{verbatim}
Similar to the calls to \texttt{moveToZeroNB()}, this block of code instructs 
the two MoBots to stand. Note that we call the non-blocking versions of the
stand function, called \texttt{motionStandNB()}. Since these functions are
non-blocking, both robots will effectively stand simultaneously. Again,
we call the \texttt{moveWait()} function on both robots to ensure that 
the robots have finished standing before the code continues.

\newpage
\appendix
\section{Data Types}
There are data types which are used by the MoBot library to represent 
certain values, such as joint id's and motor directions.

\begin{tabular}{p{3.5cm}p{7cm}} \hline 
Data Type& Description \\
\hline 
\texttt{mobotJointId\_t} & An enumerated value that indicates a MoBot joint. \\
\texttt{mobotJointState\_t} & The current state of a MoBot joint. \\
\texttt{mobotJointDirection\_t} & The current motion direction of a MoBot joint. 
\end{tabular}

\subsection{\label{sec:mobotJointId_t}\texttt{mobotJointId\_t}}
This datatype is an enumerated type used to identify a joint on the MoBot. Valid
values for this type are:
\begin{verbatim}
typedef enum mobot_joints_e {
  MOBOT_JOINT1 = 1,
  MOBOT_JOINT2 = 2,
  MOBOT_JOINT3 = 3,
  MOBOT_JOINT4 = 4
} mobotJointId_t;
\end{verbatim}
The joints are enumerated in order from one side of the MoBot to the other. Joints 1 and 4
are faceplate joints, and joints 2 and 3 are body joints.

\subsection{\label{sec:mobotJointState_t}\texttt{mobotJointState\_t}}
This datatype is an enumerated type used to designate the current 
movement state of a joint. Valid values are:
\begin{itemize}
\item \texttt{MOBOT\_JOINT\_IDLE}: This value indicates that the joint is not moving.
\item \texttt{MOBOT\_JOINT\_MOVING}: This value indicates that the joint is currently moving.
\item \texttt{MOBOT\_JOINT\_GOALSEEK}: This value indicates that the joint is currently moving
  towards a predefined goal.
\end{itemize}

\subsection{\label{sec:mobotJointDirection_t}\texttt{mobotJointDirection\_t}}
This datatype designates a MoBot joint's commanded direction of travel. Valid values
are
\begin{itemize}
\item \texttt{MOBOT\_NEUTRAL}: There is no predesignated direction for the
joint. If the joint is commanded to move to a specific location, the MoBot will
decide the best direction to move the joint in order to achieve the goal with
the smallest motion.
\item \texttt{MOBOT\_FORWARD}: Move the joint in the direction which increases its 
angular position.
\item \texttt{MOBOT\_BACKWARD}: Move the joint in the direction which decreases
its angular position.
\end{itemize}

\section{Macros}

\subsection{\texttt{MoBot\_joints\_t}}
\index{MoBot\_joints\_t}
The data type \texttt{MoBot\_joints\_t} contains the following macro datatypes.\\

\index{IMOBOT\_JOINT1}
\index{IMOBOT\_JOINT2}
\index{IMOBOT\_JOINT3}
\index{IMOBOT\_JOINT4}
\begin{tabular}{p{3cm}p{7cm}} \hline 
Value & Description \\
\hline 
\texttt{IMOBOT\_JOINT1} & Joint number 1 on the MoBot, which is a faceplate joint. \\
\texttt{IMOBOT\_JOINT2} & Joint number 2 on the MoBot, which is a body joint. \\
\texttt{IMOBOT\_JOINT3} & Joint number 3 on the MoBot, which is a body joint. \\
\texttt{IMOBOT\_JOINT3} & Joint number 4 on the MoBot, which is a faceplate joint. 
\end{tabular}

\subsection{\texttt{MoBot\_joint\_direction\_t}}
\index{MoBot\_joint\_direction\_t}
The data type \texttt{MoBot\_joint\_direction\_t} indicates the commanded direction 
of a joint on the MoBot.

\index{IMOBOT\_NEUTRAL}
\index{IMOBOT\_FORWARD}
\index{IMOBOT\_BACKWARD}
\begin{tabular}{p{3cm}p{7cm}} \hline 
Value & Description \\
\hline 
\texttt{IMOBOT\_NEUTRAL} & This value indicates automatic direction control for a joint. 
The MoBot will choose the best direction to attain the commanded joint position. \\
\texttt{IMOBOT\_FORWARD} & Move the joint in the forward direction. \\
\texttt{IMOBOT\_BACKWARD} & Move the joint in the backward direction. \\
\end{tabular}

\section{MoBotComms API}
%\lhead{libimobotcomms API Documentation}
\noindent
The header file {\bf libimobotcomms.h} defines all the data types, macros 
and function prototypes for the iMobot API library. The header file
declares a class called \texttt{CiMobotComms} which contains member functions which
may be used to control the robot.

\begin{table}[!hp]
%\capstart
\begin{center}
\caption{CiMobotComms Member Functions.}
\begin{tabular}{p{38 mm}p{77 mm}}
%\begin{tabular}{ll}
\hline
Function & Description \\
\hline
%\texttt{pose()} \dotfill & Pose multiple joints of the iMobot. \\
\texttt{CiMobotComms()} \dotfill & The CiMobotComms constructor function. This function
is called automatically and should not be called explicitly. \\
\texttt{\textasciitilde CiMobotComms()} \dotfill & The CiMobotComms destructor function. This function
is called automatically and should not be called explicitly. \\
& \\
\texttt{connect()} \dotfill & Connect to a remote iMobot module. This function connects to an already-paired iMobot module in Microsoft Windows. This function does not currently work for non-Windows operating systems, such as Mac or Linux. For those operating systems, please use the \texttt{connectAddress()} function instead. \\
\texttt{connectAddress()} \dotfill & Connect to an iMobot module by specifying its Bluetooth address. \\
\texttt{disconnect()} \dotfill & Disconnect from an iMobot module. \\
\texttt{getMotorDirection()} \dotfill & Gets a motor's currently assigned direction. \\
\texttt{getMotorPosition()} \dotfill & Gets a joint's angle. \\
\texttt{getMotorSpeed()} \dotfill & Gets a motor's speed. \\
\texttt{getMotorState()} \dotfill & Gets a motor's current status. \\
\texttt{isConnected()} \dotfill & This function is used to check the connection to an iMobot. \\
\texttt{moveWait()} \dotfill & Wait until all motors have stopped moving. \\
\texttt{pozeZero()} \dotfill & Instructs all motors to go to their zero positions. \\
\texttt{setMotorDirection()} \dotfill & Set the motor direction of a motor. Set
to "0" for automatic direction, "1" for forward, and "2" for reverse. \\
\texttt{setMotorPosition()} \dotfill & Set the desired motor position. \\
\texttt{setMotorSpeed()} \dotfill & Sets a motor's speed. \\
\texttt{stop()} \dotfill & Stop all currently executing motions of the iMobot. \\
\texttt{waitMotor()} \dotfill & Wait until the specified motor has stopped moving. \\
\hline
\end{tabular}
\end{center}
\label{mobilec_api_cbinary}
\end{table}

\section{Constants and Enumerations}
There are various macros and enumerations declared in the header file which 
represent descriptive names for otherwise meaningless values. They are 
presented in the following table and used throughout the API.

\begin{tabular}{lp{3.7in}}
Macro & Description \\
\texttt{iMobot\_motors\_e} & \\\hline
\texttt{IMOBOT\_MOTOR1} & This macro represents the first motor of an iMobot. \\
\texttt{IMOBOT\_MOTOR2} & This macro represents the second motor of an iMobot. \\
\texttt{IMOBOT\_MOTOR3} & This macro represents the third motor of an iMobot. \\
\texttt{IMOBOT\_MOTOR4} & This macro represents the fourth motor of an iMobot. \\
\texttt{IMOBOT\_NUM\_MOTORS} & This macro contains the number of motors on an iMobot. \\
 & \\
\texttt{iMobot\_motor\_direction\_e} & \\\hline
\texttt{IMOBOT\_MOTOR\_DIR\_AUTO} & Set the iMobot direction to automatic. The iMobot will choose the best direction to turn the motor in order to get to the requested position. \\
\texttt{IMOBOT\_MOTOR\_DIR\_FORWARD} & Force the motor to turn in the ``forward'' direction. \\
\texttt{IMOBOT\_MOTOR\_DIR\_BACKWARD} & Force the motor to turn in the ``backward'' direction.
\end{tabular}

\newpage
\noindent
\vspace{5pt}
\rule{4.5in}{0.015in}\\
\noindent
{\LARGE \texttt{CMobot::connect()}\index{CMobot::connect()}}\\
%\phantomsection
\addcontentsline{toc}{section}{connect()}

\noindent
{\bf Synopsis}
\vspace{-8pt}
\begin{verbatim}
#include <mobot.h>
int CMobot::connect();
\end{verbatim}

\noindent
{\bf Purpose}\\
Connect to a remote mobot via Bluetooth.\\

\noindent
{\bf Return Value}\\
The function returns 0 on success and non-zero otherwise.\\

\noindent
{\bf Parameters}\\
None.\\

\noindent
{\bf Description}\\
This function is used to connect to a mobot. The function looks inside of a 
Barobo configuration file and connects to the first mobot listed in the file.
The configuration file may be created and/or modified using the Mobot Controller
Interface, and selecting the ``Mobot $\rightarrow$ Configure Mobot Bluetooth'' menu item.

\noindent
{\bf Example}\\
Please see the example in Section \ref{sec:democode} on page \pageref{sec:democode}.\\
\noindent

\noindent
{\bf See Also}\\
\texttt{connectWithBluetoothAddress(), disconnect()}\\

%\CPlot::\DataThreeD(), \CPlot::\DataFile(), \CPlot::\Plotting(), \plotxy().\\

\pagebreak
\input{libimobotcomms_api/connectAddress}
\pagebreak
\noindent
\vspace{5pt}
\rule{4.5in}{0.015in}\\
\noindent
{\LARGE \texttt{CMobot::disconnect()}\index{CMobot::disconnect()}}\\
%\phantomsection
\addcontentsline{toc}{section}{disconnect()}

\noindent
{\bf Synopsis}
\begin{verbatim}
#include <mobot.h>
int CMobot::disconnect();
\end{verbatim}

\noindent
{\bf Purpose}\\
Disconnect from a remote robot.\\

\noindent
{\bf Return Value}\\
The function returns 0 on success and non-zero otherwise.\\

\noindent
{\bf Parameters}\\
None.\\

\noindent
{\bf Description}\\
This function is used from disconnect to a robot. A call to this function is
not necessary before the termination of a program. It is only necessary if
another connection will be established within the same program at a later time.
\\

\noindent
{\bf Example}\\
\noindent

\noindent
{\bf See Also}\\
\texttt{connect(), connectWithAddress()}

%\CPlot::\DataThreeD(), \CPlot::\DataFile(), \CPlot::\Plotting(), \plotxy().\\

\pagebreak
\noindent
\vspace{5pt}
\rule{4.5in}{0.015in}\\
\noindent
{\LARGE \texttt{CiMobotComms::getMotorDirection()}\index{getMotorDirection()}}\\
%\phantomsection
\addcontentsline{toc}{section}{getMotorDirection()}

\noindent
{\bf Synopsis}\\
\begin{verbatim}
#include <imobotcomms.h>
int CiMobotComms::getMotorDirection(int id, int &direction);
\end{verbatim}

\noindent
{\bf Purpose}\\
Get the speed of a joint on the iMobot.\\

\noindent
{\bf Return Value}\\
The function returns 0 on success and non-zero otherwise.\\

\noindent
{\bf Parameters}
\vspace{-0.1in}
\begin{description}
\item               
\begin{tabular}{p{20 mm}p{145 mm}}
\texttt{id} & The joint number to pose. \\
\texttt{direction} & An integer variable. This variable will be overwritten
with the current speed of the joint.
\end{tabular}
\end{description}

\noindent
{\bf Description}\\
This function is used to retrieve the motor's direction status. The valid
status directions are
\begin{itemize}
\item 0: Automatic direction
\item 1: Forward direction
\item 2: Backward direction
\end{itemize}

\noindent
{\bf Example}\\
\noindent

\noindent
{\bf See Also}\\
\texttt{setMotorDirection()}

%\CPlot::\DataThreeD(), \CPlot::\DataFile(), \CPlot::\Plotting(), \plotxy().\\

\pagebreak
\input{libimobotcomms_api/getMotorPosition}
\pagebreak
\input{libimobotcomms_api/getMotorSpeed}
\pagebreak
\noindent
\vspace{5pt}
\rule{4.5in}{0.015in}\\
\noindent
{\LARGE \texttt{CiMobotComms::getMotorState()}\index{getMotorState()}}\\
%\phantomsection
\addcontentsline{toc}{section}{getMotorState()}

\noindent
{\bf Synopsis}\\
\begin{verbatim}
#include <imobot.h>
int CiMobotComms::getMotorState(int id, int &state);
\end{verbatim}

\noindent
{\bf Purpose}\\
Determine whether a motor is moving or not.\\

\noindent
{\bf Return Value}\\
The function returns 0 on success and non-zero otherwise.\\

\noindent
{\bf Parameters}
\vspace{-0.1in}
\begin{description}
\item               
\begin{tabular}{p{10 mm}p{145 mm}}
\texttt{id} & The joint number to pose. \\
\texttt{state} & An integer variable which will be overwritten with the current state of the motor. 
\end{tabular}
\end{description}

\noindent
{\bf Description}\\
This function is used to determine the current state of a motor. Valid states are:
\begin{itemize}
\item 0: The motor is idle.
\item 1: The motor is moving.
\item 2: The motor is heading towards a specified position.
\end{itemize}

\noindent
{\bf Example}\\
\noindent

\noindent
{\bf See Also}\\

%\CPlot::\DataThreeD(), \CPlot::\DataFile(), \CPlot::\Plotting(), \plotxy().\\

\pagebreak
\noindent
\vspace{5pt}
\rule{4.5in}{0.015in}\\
\noindent
{\LARGE \texttt{CMobot::isConnected()}\index{CMobot::isConnected()}}\\
%\phantomsection
\addcontentsline{toc}{section}{isConnected()}

\noindent
{\bf Synopsis}
\vspace{-8pt}
\begin{verbatim}
#include <mobot.h>
int CMobot::isConnected();
\end{verbatim}

\noindent
{\bf Purpose}\\
Check to see if currently connected to a remote mobot via Bluetooth.\\

\noindent
{\bf Return Value}\\
The function returns zero if it is not currently connected to a mobot or if an 
error has occured, or 1 if the mobot is connected.\\

\noindent
{\bf Parameters}\\
None.\\

\noindent
{\bf Description}\\
This function is used to check if the software is currently connected to
a mobot.\\

\noindent
{\bf Example}\\
\noindent

\noindent
{\bf See Also}\\
\texttt{connect(), disconnect()}\\
%\CPlot::\DataThreeD(), \CPlot::\DataFile(), \CPlot::\Plotting(), \plotxy().\\

\pagebreak
\noindent
\vspace{5pt}
\rule{4.5in}{0.015in}\\
\noindent
{\LARGE \texttt{CMobot::moveWait()}\index{CMobot::moveWait()}}\\
%\phantomsection
\addcontentsline{toc}{section}{moveWait()}

\noindent
{\bf Synopsis}
\vspace{-8pt}
\begin{verbatim}
#include <mobot.h>
int CMobot::moveWait();
\end{verbatim}

\noindent
{\bf Purpose}\\
Wait for all joints to stop moving.\\

\noindent
{\bf Return Value}\\
The function returns 0 on success and non-zero otherwise.\\

\noindent
{\bf Description}\\
This function is used to wait for all joint motions to finish. Functions such as
\texttt{move()} and \texttt{moveTo()} do not wait for a joint to finish
moving before continuing to allow multiple joints to move at the same time. The
\texttt{moveWait()} function is used to wait for
mobotic motions to complete.

Please note that if this function is called after a motor has been commanded to
turn indefinitely, this function may never return and your program may hang.\\

\noindent
{\bf Example}\\
See the sample program in Section \ref{sec:democode} on page \pageref{sec:democode}.
\noindent

\noindent
{\bf See Also}\\
\texttt{moveWait(), moveJointWait()}

%\CPlot::\DataThreeD(), \CPlot::\DataFile(), \CPlot::\Plotting(), \plotxy().\\

\pagebreak
\noindent
\vspace{5pt}
\rule{4.5in}{0.015in}\\
\noindent
{\LARGE \texttt{CiMobotComms::poseZero()}\index{poseZero()}}\\
%\phantomsection
\addcontentsline{toc}{section}{poseZero()}

\noindent
{\bf Synopsis}\\
\begin{verbatim}
#include <imobot.h>
int CiMobotComms::poseZero();
\end{verbatim}

\noindent
{\bf Purpose}\\
Move all of the joints of an iMobot to their zero position.\\

\noindent
{\bf Return Value}\\
The function returns 0 on success and non-zero otherwise.\\

\noindent
{\bf Parameters}\\
None.\\

\noindent
{\bf Description}\\
This function moves all of the joints of an iMobot to their zero position.
Please note that this function is non-blocking and will return immediately. Use
this function in conjunction with the \texttt{moveWait()} function to block
until the movement completes.\\

\noindent
{\bf Example}\\
\noindent

\noindent
{\bf See Also}\\

%\CPlot::\DataThreeD(), \CPlot::\DataFile(), \CPlot::\Plotting(), \plotxy().\\

\pagebreak
\noindent
\vspace{5pt}
\rule{4.5in}{0.015in}\\
\noindent
{\LARGE \texttt{CiMobotComms::setMotorDirection()}\index{setMotorDirection()}}\\
%\phantomsection
\addcontentsline{toc}{section}{setMotorDirection()}

\noindent
{\bf Synopsis}\\
\begin{verbatim}
#include <imobotcomms.h>
int CiMobotComms::setMotorDirection(int id, int direction);
\end{verbatim}

\noindent
{\bf Purpose}\\
Set's a motor's direction. In conjunction with \texttt{setMotorSpeed()}, this
function may be used to cause a motor to turn indefinitely.\\

\noindent
{\bf Return Value}\\
The function returns 0 on success and non-zero otherwise.\\

\noindent
{\bf Parameters}
\vspace{-0.1in}
\begin{description}
\item               
\begin{tabular}{p{20 mm}p{145 mm}}
\texttt{id} & The joint number to move. \\
\texttt{direction} & A value indicating the desired direction.
\end{tabular}
\end{description}

\noindent
{\bf Description}\\
This function is used to set a motor's turn direction. Possible values for the
direction are:
\begin{itemize}
\item 0: Automatic direction. This is the default setting. 
\item 1: Forward. If this value is used with a non-zero speed set using the
\texttt{setMotorSpeed()} function, the motor will turn forward indefinitely.
\time 2: Backward. Similar to "1", except the motor will spin backward.
\end{itemize}

\noindent
{\bf Example}\\
\noindent

\noindent
{\bf See Also}\\

%\CPlot::\DataThreeD(), \CPlot::\DataFile(), \CPlot::\Plotting(), \plotxy().\\

\pagebreak
\noindent
\vspace{5pt}
\rule{4.5in}{0.015in}\\
\noindent
{\LARGE \texttt{CiMobotComms::setMotorPosition()}\index{setMotorPosition()}}\\
%\phantomsection
\addcontentsline{toc}{section}{setMotorPosition()}

\noindent
{\bf Synopsis}\\
\begin{verbatim}
#include <imobot.h>
int CiMobotComms::setMotorPosition(int id, double position);
\end{verbatim}

\noindent
{\bf Purpose}\\
Connect to a remote iMobot via Bluetooth.\\

\noindent
{\bf Return Value}\\
The function returns 0 on success and non-zero otherwise.\\

\noindent
{\bf Parameters}\\
\vspace{-0.1in}
\begin{description}
\item               
\begin{tabular}{p{10 mm}p{145 mm}}
\texttt{id} & The joint number to wait for. \\
\texttt{position} & The absolute angle to move the motor to.  \\
\end{tabular}
\end{description}

\noindent
{\bf Description}\\
This function commands the motor to move to a position specified in degrees at
the current motor's speed. The current motor speed may be set with the
\texttt{setMotorSpeed()} member function. Please note that if the motor speed
is set to zero, the motor will not move after calling the
\texttt{setMotorPosition()} function. \\

\noindent
{\bf Example}\\
\noindent

\noindent
{\bf See Also}\\
\texttt{connectAddress()}

%\CPlot::\DataThreeD(), \CPlot::\DataFile(), \CPlot::\Plotting(), \plotxy().\\

\pagebreak
\noindent
\vspace{5pt}
\rule{6.5in}{0.015in}
\noindent
{\LARGE \texttt{CiMobot::setMotorSpeed()}\index{setMotorSpeed()}}\\
\phantomsection
\addcontentsline{toc}{section}{setMotorSpeed()}

\noindent
{\bf Synopsis}\\
\begin{verbatim}
#include <imobot.h>
int CiMobot::setMotorSpeed(unsigned short id, unsigned short speed);
\end{verbatim}

\noindent
{\bf Purpose}\\
Get the speed of a joint on the iMobot.\\

\noindent
{\bf Return Value}\\
The function returns 0 on success and non-zero otherwise.\\

\noindent
{\bf Parameters}
\vspace{-0.1in}
\begin{description}
\item               
\begin{tabular}{p{10 mm}p{145 mm}}
\texttt{id} & The joint number to pose. \\
\texttt{speed} & An unsigned short variable indicating the requested speed.
\end{tabular}
\end{description}

\noindent
{\bf Description}\\
This function is used to set the speed of a joint of an iMobot. Valid speed
values range from 0 to 100.
\noindent
{\bf Example}\\
\noindent

\noindent
{\bf See Also}\\

%\CPlot::\DataThreeD(), \CPlot::\DataFile(), \CPlot::\Plotting(), \plotxy().\\

\pagebreak
\noindent
\vspace{5pt}
\rule{6.5in}{0.015in}
\noindent
{\LARGE \texttt{CiMobot::stop()}\index{stop()}}\\
\phantomsection
\addcontentsline{toc}{section}{stop()}

\noindent
{\bf Synopsis}\\
\begin{verbatim}
#include <imobot.h>
int CiMobot::stop();
\end{verbatim}

The usage of this function is identical to the
\texttt{CMobot::stop()} function for the MoBot.
Please refer to the MoBot documentation for \texttt{CMobot::stop()} for
detailed usage documentation.


\pagebreak
\noindent
\vspace{5pt}
\rule{4.5in}{0.015in}\\
\noindent
{\LARGE \texttt{CiMobotComms::waitMotor()}\index{waitMotor()}}\\
%\phantomsection
\addcontentsline{toc}{section}{waitMotor()}

\noindent
{\bf Synopsis}\\
\begin{verbatim}
#include <imobot.h>
int CiMobotComms::waitMotor(int id);
\end{verbatim}

\noindent
{\bf Purpose}\\
Wait for a joint to stop moving.\\

\noindent
{\bf Return Value}\\
The function returns 0 on success and non-zero otherwise.\\

\noindent
{\bf Parameters}
\vspace{-0.1in}
\begin{description}
\item               
\begin{tabular}{p{10 mm}p{145 mm}}
\texttt{id} & The joint number to wait for. \\
\end{tabular}
\end{description}

\noindent
{\bf Description}\\
This function is used to wait for a joint motion to finish. Functions such as
\texttt{poseJoint()} and \texttt{moveJoint()} do not wait for a joint to finish
moving before continuing to allow multiple joints to move at the same time. The
\texttt{waitMotor()} or \texttt{waitMotor()} functions are used to wait for
robotic motions to complete.

Please note that if this function is called after a motor has been commanded to
turn indefinitely, this function may never return and your program may hang.\\

\noindent
{\bf Example}\\
See the sample program in Section \ref{subsec:simple.cpp} on page \pageref{subsec:simple.cpp}.
\noindent

\noindent
{\bf See Also}\\
\texttt{moveWait()}

%\CPlot::\DataThreeD(), \CPlot::\DataFile(), \CPlot::\Plotting(), \plotxy().\\



% Index
\addcontentsline{toc}{chapter}{Index}
\printindex

\end{document}
